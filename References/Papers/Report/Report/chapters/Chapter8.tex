\chapter{Conclusions}
Unfortunately it was not able to verify the model in Chapter \ref{ch:6} with experimental data, because of sensor interference and a non-robust quadcopter that got demolished by a crash. Several things are added in the simulation which should make it realiable, i.e. thrust profiles, noises, geometry, etc. Previous research however shows that simulating a drone represents reality quite nicely. The simulation has provided information that it is possible to compensate near-wall effects with disturbance observer based control. At least, this compensation works in simulation and has to be verified in reality, which requires some extra tuning, which is explained in further detail in Chapter \ref{ch:disc}. This is quite an interesting outcome, since flying close to walls has been an issue for quite some time. The use of DOBC could be a good solution to the problem, since there is currently no existing solution to fly close to a wall in an uncontrolled environment.\\

Additionally, the theoretical route-planning has been created to fly smoothly through mine corridors. Hereby has been taken into account that there can be use of GPS underground and thereby uses another mapping/localisation technique. This theoretical routeplanning has to be fused with the mapping/localisation technique that will be used for the final version, as discussed in Chapter \ref{ch:rp}. Further research should provide the likelyhood of succes with route-planning, DOBC and obstacle avoidance combined.\\

Lastly, there has been great understanding in the quadcopter system and there are several things to take into account if flying the drone. Consequently, the report provides some Do's and Dont's to simplify the drone usage. The Kalman-filter has to be checked that the states are estimated correctly. Thereby has to be checked if there is for instance, no magnetic interference of bad GPS data.\\