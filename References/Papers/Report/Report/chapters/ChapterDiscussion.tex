\chapter{Recommendations and Discussion} \label{ch:disc}

\textcolor{red}{Add a part which says that we should make the disturbance also dependent upon the rotorspeed.}\\


Several things can be added to make the simulation more realistic and thus comparable to experimental data. This gives a benefit over tuning in the simulation instead of risking the drone to crash. Firstly, extending the model with non-linearities, which are neglected around hovering. There are neglected non-linearities in the attitude dynamics, Kalman filter and even in the near-wall disturbance. These non-linearities are however still present and give a standard error. Secondly, making controllers more robust to modeling errors by creating controllers with Robust Control. A non-linear Robust controller should even be a possibility.\\
Furthermore needs to be noted that an extra observer requires some computation time, that can overflow the OBC. Especially at the same looprate as the attitude control of $100$Hz. To get the DOBC working in reality, an addition of an external On Board Computer (OBC) will solve this issue. Some additional tuning of the DOBC to the real system or creating a robust controller in the simulation already.\\
Besides the simulation, some recommendations for the quadcopter including some additions can be done as well. First of all, and the most important one, is to buy or build a robust quadcopter that can absorb multiple crashes without being demolished. For an inexperienced dronetechnician, it will occur highly probable that the system turns unstable when testing new controllers. Hence, if the drone breaks and looses some parts or flexibility, the system also changes. This happens due to changes in mass or inertia or extra vibrations in the system (loose bolts or screws) and thereby will the controllers need to be retuned. A solution to this could be tethered flying, however this technique should be approached with caution. Tethering too loose and the drone will still be able to crash, and tethering too tight will affect your system. Thereby is the tethering that is used in Chapter \ref{ch:test}, only applicable for hovering purposes. Testing near wall-effects and position control will need other kind of tethering, if those exist.\\
Additional sensors need to be added to the quadcopter to realise the route-planning. It is recommended that the drone uses a 3D mapping method that uses SLAM \textcolor{red}{Explain in more detail!}. These method uses two gimbal cameras to map the surroundings.\\
The experimental data regarding the near-wall thrustloss has not been reliable as discussed before. Since the sensitivity of the experimental setup was not high enough, were the differences between the results hard or impossible to read. The results are now taken with a significantly high safetyfactor, and gives decent results. Controllers can be tuned more precise when the simulation is provided with a better representation of this disturbance. This means that the experiments need to be performed with a higher precision loadcell and a solid test setup.\\
Moreover the near-wall effect, it has come to the attention that it is highly dependent upon the rotorspeed, and the assumption of the propeller rotation on the speed of hovering is probably not viable. This is because the controller will boost the propeller that is close to the wall, amplifying the effects and is not taken into account so far. Consequently, these effects has to be added to the simulation to properly tune the observer gains and see if the drone still compensates these effects.\\
As discussed before, is there a conflict with the code generation that drives the platform. It is currenly the Px4v2 that runs on Ardupilot, however it is not very user friendly to make addaptions. Thereby is it recommended base the code on Simulink itself, which can be done with Hardware In the Loop (HIL) \textcolor{red}{Explain this a little bit. Or do some research, since I actually do not understand .. :)}\\




