\documentclass[12pt]{report}
\usepackage{amsmath}
\usepackage{amsfonts}
\usepackage{amssymb}
\usepackage{textcomp}
\usepackage{graphicx}
\usepackage{float}
\usepackage{hyperref}
\usepackage[a4paper, margin = 1in]{geometry}

\usepackage[draft]{todonotes}


%Commands used in text, edit here and the document will populate with correct value.
\newcommand{\projectName}{CEPAD }
\newcommand{\payLoadMass}{2Kg }

\makeatletter
\newcommand*\bigcdot{\mathpalette\bigcdot@{.5}}
\newcommand*\bigcdot@[2]{\mathbin{\vcenter{\hbox{\scalebox{#2}{$\m@th#1\bullet$}}}}}
\makeatother

%*******************************TITLE PAGE*******************************%

\title{Development of a Close Quarter Collision Protected Aerial Drone}
\author{Angus Steele} 

%*******************************START OF DOCUMENT*******************************%

\begin{document}

\maketitle

%*******************************BEGIN THE ABSTRACT*******************************%

\chapter*{Abstract}
In this work a solution to flight inside a confined environment is proposed and implementation through simulation. The goal of the design is enable a vehicle to navigate in an unknown confined space avoiding collision with walls and unexpected obstacles. The selection of an appropriate aircraft design is done, followed by a controller design and implementation of a flight strategy including an obstacle avoidance routine. The goal of the project was proven through in depth mathematical modelling of the system and the environment. Simulation and testing was performed using a model generated in Matlab and Simulink.

The best suited configuration of a quadrotor was chosen based on a rigorous analysis of existing rotorcraft, highlighting the shortcomings and benefits of each design. The craft designed needed to be streamlined to ensure maximum capabilities in a narrow space. The vehicle was mathematically modelled based on the chosen configuration and sensors that would be used.

The flight control was designed to be robust against disturbances and have tight position tracking to ensure stable flight inside an confined environment and avoid collisions. The controllers were designed and shown to be able to withstand disturbances the vehicle will be challenged of facing.

The flight strategy was developed to optimise the use of such a platform in a narrow space similar to that seen inside a mining environment. A heading alignment strategy was added to the existing controllers ensuring the drone maintains a set heading based on it's current velocity. The final consideration was a thorough design and implementation of an obstacle avoidance routine. The system had to ensure no collisions would occur and ensure suitability for the implementation of a higher route planning strategy in future work.

The simulated craft was shown to be capable of navigating inside an unknown environment and successfully avoid collisions with the use of a waypoint generator.

%*******************************DOCUMENT LISTS*******************************%

\tableofcontents
%\listoffigures
\listoftables

%*******************************SELF GENERATED GLOSSARY*******************************%

\chapter*{Glossary}
\begin{enumerate}
%\item Close-Quarter Explosion Protected Aerial Drone (CEPAD)
\item Micro Aerial Vehicles (MAV)
\item Unmanned Aerial System (UAS)
\item Unmanned Aerial Vehicle (UAV)
%\item IECEx (International Electrotechnical Commission for Explosive Atmospheres)
%\item ATEX (Atmosph\`{e}res Explosives)
\item Disk Loading (DL)
\item Power Loading (PL)
%\item South African Bureau of Standards (SABS)
%\item South African National Standards (SANS)
%\item International Organisation of Standards (ISO)
\item Degree of Freedom (DOF)
\item On Board Computer (OBC)
\item Ground Control Station (GCS)
\end{enumerate}

%*******************************CONSISTENCY RULES*******************************%
\newpage

%\todo[inline]{RULES FOR CONSISTENCY:}
%\todo[inline]{}
%\todo[inline]{Figure always with a capital F}
%\todo[inline]{Equation, use eqref (or put in brackets) and don't say equation unless it's the beginning of a sentence}
%\todo[inline]{Image citation done inline as well as "Taken from (X) in caption"}
%\todo[inline]{Equation citation done inline only}
%\todo[inline]{Section always with a capital S}
%
%\todo[inline]{}
%*******************************INCLUDE DIFFERENT CHAPTERS*******************************%

\include{Introduction}
\chapter{Literature Review}
This chapter seeks to evaluate existing research in the field of rotorcraft design and obstacle avoidance strategies. To critically review some of the high level concepts in rotorcraft design, a brief evaluation is given of flight theory and how they effect design decisions for rotorcraft. After an understanding of flight theory is obtained, it is necessary to evaluate how this theory is utilised in creating rotorcraft. Armed with a better understanding of flight generation for rotorcraft, an analysis of traditional rotorcraft configurations is completed. Due to the hazardous nature of the mission environments, existing collision protection techniques are then discussed. At this point, the reader should feel confidence that the platform design is grounded with a solid understanding of existing techniques and requirements.
Now that a platform has been developed, the next step is to review some of the methods used to control multi-rotor platforms. Once stable flight methods have been evaluated and discussed, the researcher reviewed existing methods for obstacle avoidance as well as the requirements of implementing an on-board obstacle avoidance system.

\section{Flight Theory}

Although there are many different forms of flight, each form will have a very similar force diagram as seen in Figure \ref{IM_FlightForces}. 

\begin{figure}
\centering
\includegraphics[height = 6cm]{Images/Literature/Flight}
\caption{Force diagram of basic components of flight}
\label{IM_FlightForces}
\end{figure}

From the force diagram it can be intuitively seen that to increase the velocity of a body in air, the lift must exceed the weight.
Weight is directly determined by the object's mass and the relevant gravity coefficient, $W = mg$. Lift counteracts this weight in an attempt to boost the body into the air. Upward acceleration is only achieved once lift exceeds weight, if they are equal the body will be in a state of hover. From the lift equation seen in \eqref{EQ_Lift} only a body with velocity can obtain lift. In a rotorcraft, the rotating blades move through the air and generate lift, thereby negating the requirement for the body to have any linear velocity. Unlike in a fixed wing aircraft if the vehicle is stationery, zero lift is generated, thus a vertical take-off is not possible. 

\begin{equation}
\label{EQ_Lift}
Lift = C_L(\frac{1}{2} \rho V^2) S
\end{equation}

To propel the body forward the propulsion must exceed the value of the drag force, which acts directly against its velocity vector. When no propulsion force is present, the craft will continue to lose speed due to drag. Similar to lift, drag varies with velocity as shown in equation (\ref{EQ_Drag}).

\begin{equation}
\label{EQ_Drag}
Drag = C_D (\frac{1}{2} \rho V^2) A
\end{equation}

The coefficient of drag ($C_D$) is determined by the objects shape and ultimately the way it interacts with the air flow. 

Figure \ref{IM_FlightForces} better describes a fixed wing aircraft, the lift and propulsion forces in a rotorcraft can be seen as the components of thrust which generate either a vertical or horizontal force. To better understand the design behind rotorcraft, the principles behind this thrust generation are discussed.

	\subsection{Fundamental Principles of Flow}
	Rotors generate thrust by pulling quiescent air through the rotor plane. The principles that best govern these flow dynamics and force generation are discussed below.
	
		\subsubsection{Continuity Equation and Bernoulli's Principle}
		Bernoulli observed that the mass flow in a closed system remains constant, as shown in \eqref{EQ_BernoulliP}. This principle states that in a closed system, the product of density ($\rho$), area (A) and velocity (v) for a flowing system will remain constant \cite{Dayle}. Based on this finding it was discovered that the mass flow of a flowing medium will follow the laws of continuity in the form of equation \eqref{EQ_Bernoulli}.
		
		\begin{equation}
		\label{EQ_BernoulliP}
		\Delta \rho Av = 0
		\end{equation} 
		
		The Bernoulli equation is a statement of the conservation of energies present in a flowing system \cite{Dayle}. Equation \ref{EQ_Bernoulli} considers a pipe with a flowing liquid and states that the energy will remain unchanged in a closed system. The sum of these energies will contain the kinetic energy of the liquid as well as the energy present through pressure. Bernoulli's equation can be rewritten in the form \eqref{EQ_Bernoulli2} to represent that any changes in pressure, can result in a change of the velocities.
		
		\begin{eqnarray}
		P_0 + \frac{1}{2} \rho v_0^2 &=& P_1 + \frac{1}{2} \rho v_i^2\label{EQ_Bernoulli}\\
		P_2 - P_1 &=& \frac{1}{2} \rho (v_\infty^2 - v_0^2)\label{EQ_Bernoulli2}
		\end{eqnarray}

		\subsubsection{Reynold's Number}
		Moving different objects in the same environment will create different results of flow, in the same breath moving the same object through different environments will also create various results of flow. Osborne Reynold attempted to mathematically determine these effects and quantify what caused a system to have turbulent flow opposed to laminar flow and vice versa. During his research he created a dimensionless constant known as the Reynold's number, as shown in equation \ref{EQ_Reynold} \cite{Reynold}. 
		
		\begin{equation}
		\label{EQ_Reynold}
		Re = \frac{\rho v L}{\mu}
		\end{equation} 
		
		In its simplest form, the Reynold's number of an object is the ratio between inertial forces ($\rho v L$) and viscous forces ($\mu$) of the gas. In a system where the viscous forces dominate ($Re ~< 10^3$), there will be laminar flow and when there are higher inertial forces ($Re >~ 10^4$) the flow will be turbulent. 
		Since turbulent flow will decrease stability and increase drag forces, Reynold's numbers have become a very important part in correctly modelling and designing aircraft \cite{Reynold}.

	\subsection{Basic Rotor Theory}
	The rotor is responsible for all the aspects of flight and generates the lift, forward propulsion and the means to control the orientation of the craft \cite{Leishman}. It is for this reason that an in depth understanding of rotor characteristics and performance was done. The original research into rotor analysis was done with helicopters, but the rotor theory basics are relevant to any rotating winged craft.
	
	Due to Newton's third law, any rotating blade will cause a a rotation in the opposite direction to that motion. This applied force will drive the vehicle to rotate around that spinning axis and creates the need for a counter torque mechanism. This is common with most rotorcraft and can be visualised in figure \ref{IM_Antitorque}, in the form of the tail rotor as seen in conventional helicopters. The quadrotor handles this by having an equal number of oppositely rotating rotors.
	
	\begin{figure}[H]
	\centering
	\includegraphics[height = 6cm]{Images/Literature/AntiTorque}
	\caption{Image Illustrating generation of Counter Torque (Taken from \cite{Heli})}
	\label{IM_Antitorque}
	\end{figure}
	
	The capability of any part of a rotor to produce lift is influenced by the local blade position and pressure at that point \cite{Leishman}.

	Angular velocity equations state that the speed of any part of the rotor varies along the length of the rotor. With the maximum velocity sitting at the rotor tip. As the rotor spins, the blade's angle of attack shifts. This angle is defined as an azimuth angle ($\psi$) and is measured relative to air flow. The azimuth angle is 0\textdegree down stream and sits at 180\textdegree when it faces directly upstream. As the rotorcraft adds a horizontal component to its hover or vertical flight, the relative speed of the individual rotor segments now adheres to equation \eqref{EQ_TipSpeed}.  As visualised in figure \ref{IM_TipSpeed}, the relative velocity at the any part of the rotor is affected by the azimuth angle of the blade ($\psi$), translational speed of the craft ($V_{\infty}$), angular speed of the rotor ($\Omega$) and the considered distance along the rotor blade (r) \cite{Leishman} \cite{RotorCraftHand}. 

	\begin{figure}[H]
		\centering
		\includegraphics[height = 6cm]{Images/Literature/TipSpeed}
		\caption{Velocity components of a rotor \cite{Leishman}}
		\label{IM_TipSpeed}
	\end{figure}
		
	\begin{equation}
	\label{EQ_TipSpeed}
	V_{r} = \Omega r + V_{\infty}\sin(\psi)
	\end{equation}
	
	What this relationship shows is that during forward flight the tip velocity, relative to the ground, changes even if the rotor rotates at a constant speed. This complicates the rotor dynamics at higher speeds and limits the top speed of the craft. On the retreating edge ($\psi = 270$\textdegree $\therefore sin(\psi) = -1$) if $\Omega r <= V_{\infty}$ the rotor would effectively be going backwards and the helicopter is at risk of stalling out, this is known as a stall condition \cite{Leishman} \cite{RotorCraftHand}, while the advancing edge is reaching its maximum speed by approaching Mach conditions and sever instability.

	\subsection{Momentum Theory and Thrust Basics}
	
	As mentioned above the rotors of a rotorcraft are responsible for generating all the forces that manoeuvre the vehicle. These forces are induced by pushing air through the rotor disk. With a fixed wing aircraft the analysis of the blades is simplified because the only air flow produced is from the translational velocity of the entire craft. Analysis of blade performance in a rotorcraft can be more challenging as the rotation of the blades must be considered along side the overall speed of the vehicle. As the craft manoeuvres in space, the air flow through the rotor has significant complexities which complicates the analysis. Since the rotorcraft is expected to perform in a variety of flight styles it is important to understand these models, and their flaws. 
	
	To simplify, initially consider a helicopter in a hovering state (Weight(W) = Thrust(T)). Figure \ref{IM_MomentumTheoryAirFlow}, taken from \cite{Leishman}, helps visualise the induced air flow by showing how the rotor smooths out the air by forcing it through the disk area. This more uniform air creates an edge known as the slipstream or wake boundary, with the surrounding air remaining dormant \cite{Leishman}. Inside the wake boundary, the average velocity of the air is tangible and effective, where outside the slipstream edge, the average air velocity is negligible and obsolete. The force required to push that mass of air through the disk space is, by Newton's third law, returned by the air unto the rotor. Thus giving the rotor blades a thrust component.
	  
	\begin{figure}[h]
	\centering
	\includegraphics[height = 8cm]{Images/Literature/MomentumTheoryAirFlow}	
	\caption{Visualisation of Induced Air Flow Through A Rotor \cite{Leishman}}
	\label{IM_MomentumTheoryAirFlow}
	\end{figure}
	
	Rankine-Froude's Momentum Theory looks at this induced velocity as well as the displacement of air through the propeller, and attempts to quantify the induced thrust. While figure \ref{IM_MomentumTheoryAirFlow} helps visualise the principle, the variable naming convention for the equations is shown in figure \ref{IM_MomentumTheoryHover} below. 
	Labels 0, 1, 2 and $\infty$ refer to the locations of quiescent flow, inflow directly before the rotor, airflow immediately after the disk and the slipstream\footnote{Generally far wake is considered as 1 full rotor diameter distance away \cite{Leishman}.} or far wake condition respectively.
	The velocities are shown as the induced velocity in and out the rotor ($v_{i}$), the far wake velocity ($v_{\infty}$) and finally $v_{0}$ represents the zone with zero flow rate. There is no velocity jump across the rotor, the energy being fed into the system by the rotor is represented by a pressure change.
	
	\begin{figure}[h]
	\centering
	\includegraphics[height = 8cm, angle=360]{Images/Literature/MomentumTheoryHover}			%Must show both air flow as well as far stream and such
	\caption{Momentum Theory in Hover (Adapted From \cite{Leishman})}
	\label{IM_MomentumTheoryHover}
	\end{figure}
	
	As described above, it is by forcing the air through the disk that lift is generated. The mass flow rate of this air can then be described by (\ref{EQ_MassFlow}), where ($\rho$) is the density of air and A is the area of one full blade rotation. The rate at which this mass of air is displaced becomes a crucial variable in rotor dynamics and is directly proportional to thrust (T). This relationship can be quantified as shown in (\ref{EQ_ThrustBasic}). Thrust can also be calculated by finding the difference in pressures over the rotor disk as in (\ref{EQ_ThrustPressure}). Since acceleration is the difference in $v_\infty$ and $v_0$, \eqref{EQ_ThrustMass} can also be obtained.
	
	\begin{eqnarray}
	\dot{m} &=& \rho A v_{i}\label{EQ_MassFlow}\\
	T &=& \dot{m}a\label{EQ_ThrustBasic}\\
	T &=& A(P_2 - P_1)\label{EQ_ThrustPressure}\\
	T &=& \rho A v_{i} (v_\infty - v_0)\label{EQ_ThrustMass}
	\end{eqnarray}
	
	Equation \eqref{EQ_ThrustMass} demonstrates the negative effect of having active ambient air. This could be caused by surrounding turbulent airflow, wind or even translational movements and will need to be considered in the controller design.

	\subsection{Disk and Power Loading}
		\subsubsection{Disk Loading}
		Disk loading (DL) is a term seen often in the world of rotorcraft, it is a simple but important ratio between thrust and the area a rotating disk makes. It is represented in \eqref{EQ_DL}. Since the pressure drop across each rotor is considered uniform, the disk loading for each rotor will equate to the pressure drop across that disk.
		 
		\begin{equation}
		\label{EQ_DL}
		DL (\frac{N}{m^{2}})= \frac{T}{A} = \frac{1}{2} \rho v_\infty^2
		\end{equation}
		
		For multi-rotor craft, the disk loading is assumed uniform across all rotors \cite{Leishman}. The overall disk loading of a single rotorcraft such as a traditional helicopter will be lower than that of a multi-rotor craft of a similar size \cite{RotorCraftHand}.  Figure \ref{IM_DL} shows some examples of disk loading values for a variety of rotor configurations, as shown disk loading is also a measure of hover efficiency.
		\todo{Edit this graph to include multirotors}
		\begin{figure}[H]
		\centering
		\includegraphics[height = 6cm]{Images/Literature/DL}     
		\caption{Image representing, various Disk Loading values for varying rotorcraft (Taken from \cite{Leishman})}
		\label{IM_DL}
		\end{figure}
		
		A higher disk loading value results in larger values for induced velocities as well as the required power to hover. This means that the larger the blades, the better the efficiency. More force will be generated by pushing large quantities of air slowly, than forcing small amounts of air through at high speeds. Of course with bigger blades, comes larger rotational inertia and geometry as well as the craft being less immune to gusts and interferences. A larger blade also creates faster tip velocities, which will limit the speed of the craft severely \cite{Leishman}.
		
		
		
		\subsubsection{Power Loading}
		Power is given by the product of both Thrust and the induced velocity at the blade. It can be written as shown in equation (\ref{EQ_Power}). What this ratio shows is that the ideal power is in cubic proportion to the induced velocity at the rotor. Therefore to reduce required power the rotor's induced velocity must be small, which can be accomplished by a significant increase in disk area \cite{Leishman}.
		
		\begin{equation}
		\label{EQ_Power}
		P = 2 \rho A v_{i}^3
		\end{equation}
		
		Another important ratio is between thrust and power, it is called power loading (PL) and is shown in equation (\ref{EQ_PL}). Power loading can be seen as a measure of craft efficiency. 
		
		\begin{equation}
		\label{EQ_PL}
		PL (\frac{N}{kW})= \frac{T}{P}
		\end{equation}
		
		From equations \eqref{EQ_DL} and \eqref{EQ_PL} it can be shown that power loading is inversely proportional to disk loading. Therefore a craft with a lower disk loading will generally be a more efficient platform.

	\subsection{Electrical Power to Thrust}
	Equation (\ref{EQ_Power}) gives a quantitative approach to solving for aerodynamic power ($P_i$). If electrical power is taken as $P_e = VI$, where V is the applied voltage and I is the sourced current, with an efficiency of $\eta$ then $P_i = \eta VI$. Noting that $P_i = T v_i$ and using equation (\ref{EQ_Power}), a relationship between thrust and $P_e$ can be formed and is represented in equation (\ref{EQ_ElectricalPowerThrust}).
	
	\begin{equation}
	\label{EQ_ElectricalPowerThrust}
	T = (2\rho A)^{\dfrac{1}{3}} (\eta P_e)^{\dfrac{2}{3}}
	\end{equation}
	
	Equation (\ref{EQ_ElectricalPowerThrust}) brings to light a very important relationship which states that thrust grows at a slower rate than the electrical input power to the system.
	
	\begin{equation*}
	T \propto P_e^{\dfrac{2}{3}}
	\end{equation*}

\section{Analysis of Conventional Rotor Wing Configurations}\label{SECT_RotorConfig}

Some of the fundamental theories described above relate to the basics behind various rotor configurations and even varying flight techniques. Each different arrangement of blades introduces certain advantages and disadvantages to the system. Not every configuration will be applicable for all operations and it is important to determine what criteria are critical for the intended application.  An analysis of varying rotor configurations is done below and follows a similar trend to that seen in \cite{RotorConfig}, \cite{Bohorquez} and \cite{NewMAV}. The main weighted criterion for the discussion were listed in no particular order as:

\begin{enumerate}
	\item Flight time and efficiency
	\item Geometry and size
	\item Drone Manoeuvrability
	\item Control algorithms
	\item Mechanical complexity
\end{enumerate}

Before the analysis can be done, certain operating parameters of the different craft, surrounding the above mentioned criteria, need to be understood. There has been discussions regarding why and how rotor blades produce lift, this section discusses the real world implementation of those blades.


The same way that a car tyre is the only way the energy from the engine is translated into motion, the rotor in a rotorcraft is responsible for taking the kinetic energy from the motors and translating it into flight. Typically a rotorcraft will be designed with either fixed pitched, or variable pitched rotors. A fixed pitched rotor is a rotor that has an optimally selected, unchangeable pitch and therefore a fixed angle of attack. This of course means that since the angle of attack is fixed for the blade, an increase in speed will be required for a change in lift. With a variable pitched blade, the pilot can change the angle of attack to increase the forces. As the angle of attack increases, the blade will produce more lift without changing the speed of the motor. However, as the pitch increases, so does the drag of the blade. This then requires more motor power to keep the blade moving through the air.
The power requirements for either system are fairly similar, the advantages of a varying pitch is a single rotor has the potential for more dynamic force applications. The downfall however is the high level of complexity in the mechanical design. Both of these factors become pertinent in the final decision making of the platform design.

It is also known that any rotating member will produce a counter rotating torque to the static body, which means that any system with only one rotor will have inherent instability in the yaw axis. The end goal is to have a craft that can fly stably and accurately in 3 dimensions.

Having only a single, fixed pitched rotor allows only for control in the amount the craft flies up or down, as well as this fore mentioned instability. There are many different methods to obtain full six degrees of flight freedom. The following discussion tries to address each point listed above for different traditional methods, while trying to achieve an optimised design.


\subsection{Helicopter}
A conventional helicopter is still the most widely used configuration for large rotorcraft \cite{RotorConfig}. It consists of a single main rotor, coupled with a smaller counter rotating rotor located in the tail as seen in figure \ref{IM_Helicopter}\footnote{(Adapted from \cite{Heli})}, this is to counteract the developed counter torque as shown in Figure \ref{IM_Antitorque}.

\begin{figure}[H]
	\centering
	\includegraphics[height = 6cm]{Images/Literature/MainHeliComponents}     
	\caption{Main Components of a helicopter (Taken from \cite{Heli})}
	\label{IM_Helicopter}
\end{figure}

The main rotor of a standard helicopter has very low disk loading which gives it excellent hover efficiency. Since the desired end product will mostly be in a state of hover or at least slow lateral movement, this will yield good results for flight time. To achieve yaw stability this configuration makes use of a small tail rotor to counter act the induced moments. The extended tail rotor requires energy which it will draw from the motor while also adding a significant amount of length and weight to the craft. Since the single rotor only gives the pilot thrust control and the tail rotor gives measurable yaw control, there is need for more control surfaces to do more manoeuvring. To implement this most helicopters use a variable pitched rotor system. Cyclic control of this pitch allows the pilot to adjust the angle of attack of the rotor blades while they rotate, thus a forward pitch can be applied by increasing the lift on the left\footnote{This is true for an American style helicopter. The French design requires am increase of lift to the right}. This set up is mechanically very complex and takes intensive control algorithms and laws to give stable control.

Even though the classic helicopter image is always seen as a main rotor with a smaller rotor at the tail, there are many different types of anti torque tail set ups. The ducted fan approach increases the efficiency of the tail rotor by channelling the air flow of the rotor. The NOTAR design \cite{US4200252} as seen in Figure \ref{IM_NOTAR} manipulates the airflow generated by the main rotor and directs it to counter act the induced torque. A tip-jet design eliminates the torque applied to the airframe and therefore no tail rotor is required \cite{RotorConfig}. 

\begin{figure}[H]
	\centering
	\includegraphics[height = 6cm]{Images/Literature/NOTAR}     
	\caption{Image demonstrating the NOTAR system (Taken from\cite{Heli})}
	\label{IM_NOTAR}
\end{figure}

There have been many attempts at improving the standard helicopter design. These improvements have taken the form of adding rotors, designing hybrid aircraft and complex mechanical designs to harvest advantages of both the fixed wing and VTOL craft. Some have even tried to combine multiple features as Flanigan \cite{US7147182} did in his design of a tip-jet, compound, tilt rotor aircraft. 
In an attempt to keep the mechanical complexity to a minimum, not all configurations were investigated.

\subsection{Coaxial Rotors}
A coaxial configuration consists of two counter rotating blades located about the same centre of rotation that both use the same drive system. This eliminates the need for a tail rotor as the torque applied by both rotors cancel each other out, as shown in figure \ref{IM_Coaxial}.  

\begin{figure}[H]
	\centering
	\includegraphics[height = 6cm]{Images/Literature/Coaxial}     
	\caption{A standard Coaxial rotor set up and the induced forces}
	\label{IM_Coaxial}
\end{figure}

Localising the blades around a single point helps with the geometry of the craft as it is a more compact design. Using fixed pitched rotors, this platform will only give yaw and over all thrust control. Bohorquez et al in \cite{Bohorquez} attempted a number of lateral control methods, eventually settling on aerodynamic flaps to control the flow of the downwash, that and other methods are shown in figure \ref{IM_Coaxial_Variations}. Briod et all also used the same set up in his team's design of the Gimball \cite{Briod2012}.

\begin{figure}[H]
	\centering
	\includegraphics[height = 3cm]{Images/Literature/Coaxial_Configs}     
	\caption{Different methods of lateral control in a Coaxial MAV (Adapted from \cite{Bohorquez})}
	\label{IM_Coaxial_Variations}
\end{figure}

The control flaps are the most common used form of lateral control for small coaxial MAVs. They introduce little mechanical complexity and do not require excessive power to use. The flaps do however decrease efficiency of the system by interfering with the rotor airflow. If designed correctly the flaps should only influence the system while in use. For hover and vertical flight the impact will be negligible. As a control surface the flap is quite rudimentary and will require some more advanced control methods as well as in depth testing to obtain smooth flight transitions. Due to its compactness the design can have considerable manoeuvrability if the control algorithms are designed effectively. Each flap will require an actuator, this will increase total weight, power consumption and required mechanics \cite{Bohorquez}. 

The overlap of the rotors also induces an inefficiency into the system. Johnson in \cite{HeliTheory}, says there is much debate in how the loss of power is calculated. He states two of his preferred methods, the method chosen has a better approximation for small overlaps and is shown in \eqref{EQ_OverlapEfficiency} \cite{HeliTheory}. $\Delta P$ is the interference power (considered here as a fraction of total power) and $m$ is the overlap fraction and is calculated in \eqref{EQ_Overlap} \cite{HeliTheory}.

\begin{equation}
\label{EQ_OverlapEfficiency}
\frac{\Delta P}{P} = (\frac{2}{2-m})^{1/2} - 1
\end{equation}

\begin{equation}
\label{EQ_Overlap}
m = \frac{2}{\pi} \Bigg[ \cos^{-1}\frac{l}{2R} - \dfrac{l}{2R}\sqrt{1 - {\dfrac{l}{2R}}^2} \Bigg]
\end{equation}

These quantities assume a small vertical separation so that the inflow rates of both rotors can be considered the same. To calculate the overlap function, the rotor radius $R$ is needed as well as the separation distance $l$. 

\subsection{Tandem Rotors}

A tandem rotorcraft is sometimes referred to as a dual rotor, as it consists of two blades to generate thrust and thereby decreasing disk loading and increase the lift capacity. In a tandem configuration the blades sit in the front and the rear of the craft. Tandems are often used in applications that require heavier loads than the traditional rotorcraft can effectively offer. A tandem configuration is demonstrated in Figure \ref{IM_Tandem}\footnote{Image taken from https://www.snafu-solomon.com/2011/11/ch-46-flight-ops-aboard-uss-new-orleans.html}, the blades spin in opposite directions to counteract the other's rotational torque. Pitch and Yaw control are readily available through manipulation of the rotor speeds, while roll control is not as easily accomplished with this design and generally require variable pitch rotors \cite{Oh2005}. Using two smaller blades also decreases the effects of interferences such as gusts on the craft. 

\begin{figure}[H]
	\centering
	\includegraphics[height = 6cm]{Images/Literature/Tandem}     
	\caption{A military CH-46E Sea Knight, example of a typical tandem rotor}
	\label{IM_Tandem}
\end{figure}


As described in \eqref{EQ_ElectricalPowerThrust} the thrust of the system increases slower than the electrical power input into the system. In a standard configuration, doubling the electrical power would only increase the thrust by a factor of $\approx 1.587$. Where as doubling the amount of rotors being driven will double both the thrust and the electrical power. This gives the tandem arrangement the capability of lifting heavier loads with relatively low power consumption, as well as demonstrating low power consumption for hover and slow translatory flight. Having twin blades does increase the size of the craft, but the elimination of the tail rotor sees the size being similar to that of a classic helicopter.

\subsection{Multirotor Designs}
Drones have joined other remote controlled vehicles in the world of hobbyists. Of all the different designs the multirotor is the most popular. The four rotor design is generally chosen due to its incredible stability and manoeuvrability. Similar to the tandem, quadrotors have very good disk loading and thus can be used to lift heavy loads, there are even products that have 8 rotors to seriously increase the payload capability. This does however relate to a more power hungry system and a less efficient hover.

As shown in Figure \ref{IM_CounterBlades}, a quad rotor consists of two pairs of counter rotating propellers. Each shaft will be driven by its own motor and will contribute to the overall thrust and moment generation of the craft. Having the freedom to control each blade independently gives the pilot advanced manoeuvrability, with minimal mechanical complexities. This also reduces the complexity of the control algorithms as six degrees of freedom can be obtained by simply adjusting the speed of the motors. Besides the poor hover efficiency, the biggest downside of the multirotor designs is their size and weight. Each blade requires a drive system and space to rotate without interference. This generally limits the flight time of multirotors.

\begin{figure}[H]
	\centering
	\includegraphics[height =6cm]{Images/Literature/quadrotor}
	\caption{Quadrotor configuration \cite{ThrustCritical}}
	\label{IM_CounterBlades}
\end{figure}

\subsection{Tilt Rotors}
A tilt rotor is a very sophisticated system that attempts to harness the benefits of both the fixed and rotor wing aircraft. With the addition of a pivoting axis for each blade the craft has the forward flying speeds of a fixed wing craft while still being able to take off an land vertically like a rotorcraft. The tilt rotor's major downfall is related to the required highly complex and intricate mechanical design \cite{RotorConfig}.

VTOL applications require a larger blade to decrease the disk loading, while in forward flight a smaller diameter blade is desired to increase the efficiency of propulsion. Hager \cite{US6030177} developed a telescopic system that transforms the blades to get the optimal benefits out of each configuration, shown in figure \ref{IM_TiltRotor}\footnote{(Taken from \cite{Heli})}. These and other improvements have established the tilt rotor as a competitive design in the field of aeronautic transportation \cite{RotorConfig}.

\begin{figure}[H]
	\centering
	\includegraphics[height = 6cm]{Images/Literature/TiltRotor}     
	\caption{Hager's design for a telescopic tilt rotor system \cite{Heli}}
	\label{IM_TiltRotor}
\end{figure}

\subsection{Discussion}


\section{Quadrotor Flight Dynamics}

This section will discuss some of the methods and limitations pertaining to modelling the flight dynamics of a rotorcraft. Most of the discussion will surround multirotors, specifically quadrotors, as the majority of the literature is based on these designs \cite{Luukkonen, RealTime, Pounds2006, Hoffmann, Moller2015}. Due to the mechanical complexity of swashplate designs, the discussion is assuming a fixed pitched rotor set up. 

Before control laws can be applied there must be a dynamic model of the craft. To create the model there must be a good understanding of the factors that effect these dynamics as well as the mathematical methods for deriving the equations. A brief introduction to the nomenclature and axis systems is done and is followed by a discussion into modelling rotorcraft forces and moments. After the model can be obtained mathematically it is important to discuss the physical implementation of obtaining the data, and the instrumentation required. Unfortunately its very rare to have a flying environment that is void of disturbances, this section is closed with a discussion about the various disturbances that effect the flight dynamics of rotorcraft, including some specific environmental disturbances.

	\subsection{Coordinate Systems, Rotations and Nomenclature}
	As the rotorcraft manoeuvres through space, two separate sets of axes are created. Each axis system is important and transforming easily between these frames is necessary. Some of these methods are described in this section, as well as the various means of representing these rotations. This section begins by describing these different frames, namely the inertial and body frames.
		
		\subsubsection{Inertial and Body Frame}
		The inertial, or North East Down (NED), aligns itself with the North and East directions on a compass. The third axis will align with gravity as a positive Z component. This frame assumes that the earth is flat and non-rotating and this frame's origin can be defined arbitrarily.
		
		The body frame aligns itself with the body of the drone, with the nose of the craft facing in the positive X direction and the Z axis is defined perpendicular to the rotor plane with thrust generated in a negative Z direction. The origin of the body frame is defined as the center of mass for the drone.
		
		Figure \ref{IM_Frames} is a visual representation of both frames.
		
		\begin{figure}[H]
			\centering
			\includegraphics[height = 6cm]{Images/Literature/AxesOverlay.jpg}     
			\caption{The inertial and body frames}
			\label{IM_Frames}
		\end{figure}
			
		In order to relate the motion of the craft in the body frame to the inertial frame, it is necessary to be able to represent the rotation between these frames.
		
		\subsubsection{Euler Angles}	
		The most intuitive way to represent the rotation between two frames, is by looking at the rotation between each corresponding axis. These are known as the Euler angles and are made up of a roll ($\phi$), pitch ($\theta$) and yaw ($\psi$) angles. Euler angles provide a very intuitive understanding of the rotation between the different frames. This is best explained with Figure \ref{IM_Euler}.
		
		\begin{figure}[H]
			\centering
			\includegraphics[height = 4cm]{Images/Literature/rotations.jpg}     
			\caption{Individual rotations around the X, Y and Z axes respectively.}
			\label{IM_Euler}
		\end{figure}
		
		The yaw angle is defined as a pure rotation around the Z-Axis. Roll and pitch are defined as pure rotations around the X-Axis and Y-Axis respectively. The Euler angle representation does have limitations, such that any 3 Euler angles could represent a different rotation, based on the order it is applied. For this project, a Euler 3-2-1 sequence will be followed. There is also a chance of a singularity at extreme angles, this is not a concern for this project, as it will only be necessary to ever complete small rotations \cite{quaternion, Moller2015}. 
		
		According to Euler's theory, any two varying coordinate axes can always be related to one another by a single rotation. 
			
		\subsubsection{Direct Cosine Matrix}
		The direct cosine matrix (DCM), provides a simple method for transforming references between two different frames. This is necessary for converting the NED frame to the body frame and vice versa. The DCM is calculated by following 3 individual rotations and multiplying their results together. A 3-2-1 Euler sequence will transform first using yaw then pitch and finally roll. Each transformation is represented as a 3x3 Matrix representing a rotation around one of the axes.
		
		In the case of rotating from the body to the NED frame, the matrix takes the form as shown in equation \eqref{EQ_RotationMatrix} \cite{Luukkonen, Moller2015} where $C_x = \cos(x)$ and $S_x = \sin(x)$. The matrix is also orthogonal, which means that $\textbf{R}^{-1} = \textbf{R}^T$. $\textbf{R}^T$ would be the rotation from the inertial frame to the body frame \cite{Luukkonen, Moller2015, quaternion}.
		
		\begin{equation}
		\label{EQ_RotationMatrix}
		\textbf{R} = 
		\begin{bmatrix}
		C_\psi C_\theta   	& C_\psi S_\theta S_\phi - S_\psi C_\phi & C_\psi S_\theta C_\phi + S_\psi S_\phi \\
		S_\psi C_\theta   	& S_\psi S_\theta S_\phi + C_\psi C_\phi & S_\psi S_\theta C_\phi - C_\psi S_\phi\\
		-S_\theta   		& C_\theta S_\phi & C_\theta C_\phi  \\
		\end{bmatrix}
		\end{equation}
		
		The DCM does provide a mathematically simple method for creating relationships between frames, however this method is computationally taxing as it is forced to recalculate the matrix and the multiplications on every loop.
		
		\subsubsection{Quaternions}
		The quaternion representation manages to minimise the computation required to calculate transformations, as well as remove the singularity found in the Euler representation \cite{quaternion}. One of the major downsides of quaternions is that they are difficult to interpret intuitively. A quaternion follows the form seen in \eqref{EQ_Quaternion} and contains a scalar value $q_w$ and a vector component $[q_x \ q_y \ q_z]$. This representation is broken up into a rotation angle, and a rotation axis.
		
		\begin{equation}
		\label{EQ_Quaternion}
		\textbf{q} = 
		\begin{bmatrix}
		q_w \\
		q_x\\
		q_y\\
		q_z\\
		\end{bmatrix}
		\end{equation}
		
		Quaternions come with their own set of mathematical rules and laws which will not discussed here. However it should noted that their are techniques that provide simple conversion from and to Euler angles and thus the DCM.  
		
		\subsubsection{Nomenclature}
		The naming convention used, follows Moller's notation \cite{Moller2015} and is shown in \ref{TAB_Nomenclature}. It makes sense that the global position and velocity of the craft be described in the NED frame, however the forces and moments will be generated in the body frame. Since there is now a simple relationship between the two frames, it is possible to relate the body frame forces and movements, into earth frame translations. The variables shown in Table \ref{TAB_Nomenclature} are visualised in Figure \ref{IM_Nomenclature}. The variables are all defined in the body frame and are shown, along with their positive directions. The right hand and thumb rule were used to dictate direction.
		
		\begin{figure}[H]
			\centering
			\includegraphics[height = 10cm]{Images/Literature/ForcesMoments.jpg}     
			\caption{Typical naming convention of body forces, moments and velocities for a quadrotor}
			\label{IM_Nomenclature}
		\end{figure}
		
		\begin{table}[!]
			\centering
			\begin{tabular}{| l | l |}
				X, Y, Z 	& Force vector along the respective body frame axis\\
				L, M, N 	& Moment around the respective body frame axis\\
				U, V, W 	& Linear velocity along each body frame axis\\
				P, Q, R  	& Angular velocity around each body frame axis\\
			\end{tabular}
			\label{TAB_Nomenclature}
			\caption{Standard Nomenclature}
		\end{table}
		
		The body frame forces, moments and velocities can be seen, and are described in \eqref{EQ_ForcesMomentsVelocitiesX} - \eqref{EQ_ForcesMomentsVelocitiesN}. Where X, Y, Z are the forces in each body axis, with the rotor thrust being produced in the negative Z direction. L, M, N are the moments around the x, y, z axes respectively and U, V, W are the velocities aligned with the x, y, z axes respectively. 
		
		Using the rotation matrix described in \eqref{EQ_RotationMatrix}, a relationship for North, East and Down velocities can be made and is described in \eqref{EQ_UVWtoNED}. 
		
		\begin{equation}
		\begin{bmatrix} \dot{N}\\ \dot{E}\\ \dot{D} \end{bmatrix} = \textbf{R} \begin{bmatrix} U\\ V\\ W \end{bmatrix}
		\label{EQ_UVWtoNED}
		\end{equation}
				
	\subsection{Kinetics and Kinematics}
	
	Assuming the system can be considered as a rigid body, allows for the use of normal Newtonian mechanics to create the equations of motion. This method will also use the Euler angles described above \cite{Luukkonen, Modelling, Moller2015}.
	
	The derivations of these calculations are well documented in literature \cite{Blakelock}
	
	\begin{eqnarray}
	X &=&  m(\dot{U} - VR + WQ)\label{EQ_ForcesMomentsVelocitiesX}\\
	Y &=&  m(\dot{V} - UR + WP)\label{EQ_ForcesMomentsVelocitiesY}\\	
	Z &=&  m(\dot{Q} - UQ + VP)\label{EQ_ForcesMomentsVelocitiesZ}\\
	L &=&  \dot{P}I_{xx} + QR(I_{zz} - I_{yy})\label{EQ_ForcesMomentsVelocitiesL}\\
	M &=&  \dot{Q}I_{yy} + PR(I_{xx} - I_{zz})\label{EQ_ForcesMomentsVelocitiesM}\\	
	N &=&  \dot{R}I_{zz} + PQ(I_{yy} - I_{xx})\label{EQ_ForcesMomentsVelocitiesN}
	\end{eqnarray}
	
	
	\subsection{Mass Model and the Inertia Tensor}
		\subsubsection{Mass Model}
		In any aerial vehicle mass is always an important design criterion. Every aspect of the platform must be designed to be the lightest it possibly can. Having a light weight craft is one part of the design criterion, another would be ensuring that the weight is geometrically spread out correctly, as well as functionally distributed appropriately. The table below was adapted from \cite{NewMAV} and demonstrates the latter point by showing the weight distribution of three separate crafts. Depending on the different criteria for the craft, different functional blocks will be allocated a certain percentage of weight. For example if the project requires a longer flight time, a higher percentage would be given to the power source and possibly less to the external payload. Generating a good mass model before designing helps better understand the requirements for the craft and could be a deciding factor in the construction.
		
		\begin{table}[H]
			\centering
			\begin{tabular}{l | c | c | c | c}
				
				Component 					& 0.3kg & 1.8kg & 3.7kg\\
				\hline\hline
				Rotor System 				& 11.0 & 11.2 & 13.9\\
				Tailboom Assembly 			& 8.0 & 9.1 & 7.8\\
				Main Rotor Motor 			& 15.4 & 10.5 & 8.1\\
				Fuselage/Structure 			& 7.0 &  15.1 & 12.0\\
				Main Transmission 			& 2.0 &  3.4 & 3.4\\
				Landing Gear 				& 2.3 &  3.4 & 2.9\\
				Control System 				& 5.7 & 18.3 & 9.3\\
				Avionics 						& 29.4 &  2.4 & 1.6\\
				Power Source 				& 19.2 & 26.6 & 41.0\\
				
			\end{tabular}
			\caption{MAV Weight Data (Adapted from \cite{NewMAV})}
		\end{table}
	
		\subsubsection{Inertia Tensor}
		It was also mentioned that the weight needs to be geometrically positioned correctly, the point of this would be to create as much symmetry in the craft as possible. If this is done correctly the principle axes of inertia will align very closely with the body of the craft, simplifying calculations later on and helping find and define the principle axes. The inertia tensor is a matrix that is a representation of a rigid body's resistance to rotations in 3D space. For the general case the inertia tensor takes the form as shown in equation \eqref{EQ_InertiaTensor}. The inertia tensor is very dependant on a craft's symmetry, and is symmetric itself. In other words, $I_{xy} = I_{yx}$, $I_{xz} = I_{zx}$ and $I_{zy} = I_{yz}$ and therefore if a craft is symmetric about the X, Y and Z axes, then the assumption can be made that $I_{xy} = I_{yx} = I_{xz} = I_{zx} = I_{yz} = I_{zy} = 0$ \cite{Luukkonen, MiniFlying}.
		
		\begin{equation}
		\label{EQ_InertiaTensor}
		\textbf{I} = 
		\begin{bmatrix}
		I_{xx}	& -I_{xy} & -I_{xz}\\
		-I_{yx}	& I_{yy}	& -I_{yz}\\
		-I_{zx}	& -I_{zy}	& I_{zz}\\
		\end{bmatrix}
		\end{equation}
		
		In order to successfully model a rotorcraft, the inertia tensor must be known and will be defined around the centre of rotation of that rotorcraft. The method for obtaining the inertia tensor is described in the system identification of this project.
		
	\subsection{Rotor Generated Forces and Moments}
	The forces and moments generated by the rotors are discussed here. It is assumed that the rotors will only generate a force perpendicular to their plane while the moments are dependant on the placement of the rotors.
	
	\begin{figure}[H]
		\centering
		\includegraphics[height = 10cm]{Images/Literature/quadforces.jpg}     
		\caption{Forces and moments acting in the body frame on an X-Configuration quadrotor}
		\label{IM_Forces}
	\end{figure}
	
	
		\subsubsection{Actuators}
		As shown in Figure \ref{IM_Forces}, all the forces generated by the quadcopter are a product of the four rotors. The rotors convert mechanical energy from the motors into aerodynamic power. The motors convert electrical energy into mechanical energy based on the motor commands sent from the controller. Both the motors and rotors can not react instantly to new commands, this lag introduces a timing delay constant into the system \cite{Moller2015}.
		
		If the lag timing constant is defined as $\tau$, thrust generated by motor $x$ as $T_x$ and the command sent to that motor as $T_{xR}$. Then \eqref{EQ_MotorDelay} can be created and applies to all four motors.
		
		\begin{equation}
		T_x = -\dot{T_x} \tau + T_{xR}
		\label{EQ_MotorDelay}
		\end{equation}
		 		
		\subsubsection{Controlled Body Forces}
		Figure \ref{IM_Forces} assumes that all of the rotors lie in the same plane, and only provide a unidirectional force. This assumption allows the easy creation of a total force $\textbf{T}$, which is shown in \eqref{EQ_Translational} as the sum of all four motor thrusts. 
		
		\begin{eqnarray}
		Z = (T_1 + T_2 + T_3 + T_4)
		\label{EQ_Translational}
		\end{eqnarray}
		
		To command this value, a virtual actuator can be created $\delta_{Z}$ which commands all four rotor thrusts. Equation \eqref{EQ_MotorDelay} demonstrates the lag to generate these thrusts and the same lag dynamics will apply to $\delta_{Z}$, thus creating \eqref{EQ_VirtualHeave}.
		
		\begin{eqnarray}
		Z = -\dot{Z} \tau + \delta_Z
		\label{EQ_VirtualHeave}
		\end{eqnarray}
				
		\subsubsection{Controlled Body Moments}
		A quadrotor generates a moment around it's own axis through varying the speed of each motor. The torque generated is also dependant on the spacing for the type of quadrotor used. A standard X-Configuration quadrotor is shown in Figure \ref{IM_Forces} and was used for this analysis. To induce a torque around the X-axis, the sum of the two left rotors subtracted from the sum of the two right rotors must be non-zero. Similarly the front and back rotor summations must not be equal to induce a torque around the Y-axis. As shown in \ref{IM_Forces}, each rotor also creates a moment around the Z-axis. This induced torque is a product of the rotors lift to drag ratio and the chord length and is represented in \eqref{EQ_YawTorque}.
		
		\begin{equation}
		\tau_{\psi x} = \frac{r_D}{R_{LD}} \times T_x
		\label{EQ_YawTorque}
		\end{equation}
				
		Assuming that each rotor has the same characteristics and are spaced evenly, these moments can be mathematically expressed as shown in \eqref{EQ_Torques}, where $l$ is the distance from the centre of the rotor to the centre of gravity, $r_D$ is the chord length and $R_{LD}$ is the lift to drag ratio for the rotors.
				
		\begin{eqnarray}
		L &=& \frac{r_D}{R_{LD}} \times (T_3 + T_4 - T_1 - T_2)\\
		M &=& (T_1 + T_3 - T_4 - T2) \times lcos(\alpha)\\
		N &=& (T_2 + T_3 - T_1 - T4) \times lsin(\alpha)
		\label{EQ_Torques}
		\end{eqnarray}
		
		Virtual actuators can be created to command these moments, namely $\delta_\psi$, $\delta_\theta$ and $\delta_\phi$. However these commands will be subject to the same time delay experienced by the rotors. Therefore \eqref{EQ_VirtualTorquesL} - \eqref{EQ_VirtualTorquesN} can be used to represent the relationship between these commanded values and the actual moment \cite{Modelling, Moller2015}.
			
		\begin{eqnarray}
		L &=& -\dot{L} \tau + \delta_\psi\\ \label{EQ_VirtualTorquesL}
		M &=& -\dot{M} \tau + \delta_\theta\\ \label{EQ_VirtualTorquesM}
		N &=& -\dot{N} \tau + \delta_\phi \label{EQ_VirtualTorquesN}
		\end{eqnarray}
				
	
	\subsection{Disturbances}\label{SSECT_Disturbances}
	
		\subsubsection{Drag}
		Drag is a damping force whose quantity is relative to the speed of the object, and always opposes the direction of motion. Drag is defined here in the body frame and from \eqref{EQ_Drag} and the discussion drag, the equations for three dimensional drag can be calculated. As shown in \eqref{EQ_dragForcesX} - \eqref{EQ_dragForcesZ}, the effect of drag can be reduced through mechanical design and flight strategy, by reducing the area of the plane facing towards the direction of the motion. 
		
		\begin{eqnarray}
		F_{Dx} &=& C_{DX} (\dfrac{1}{2} \rho U^2) A_{YZ} \label{EQ_dragForcesX}\\
		F_{Dy} &=& C_{DY} (\dfrac{1}{2} \rho V^2) A_{XZ} \label{EQ_dragForcesY}\\
		F_{Dz} &=& C_{DZ} (\dfrac{1}{2} \rho W^2) A_{XY} \label{EQ_dragForcesZ}
		\end{eqnarray}
		
		Due to an offset between the centre of gravity and the centre of pressure, the drag forces can also create undesired moments. Equations \eqref{EQ_dragMomentsL} - \eqref{EQ_dragMomentsL} can be derived from the Figure \ref{IM_dragForces}, where $d_x, d_y, d_z$ are the offsets of the centre of pressure. $F_{Dx}, F_{Dy}, F_{Dz}$ are the forces generated by drag act along the coinciding body axis. $M_{Dx}, M_{Dy}, M_{Dz}$ are the and moments generated by the drag forces and the offset of the centre of pressure, they act around the coinciding axis. $A_{YZ}, A_{XZ}, A_{XY}$ are the surface areas facing the corresponding plane in the body frame with $C_{DX}, C_{DY}, C_{DZ}$ as the corresponding drag coefficients.
		
		\begin{eqnarray}
		M_{Dx} = F_{Dz} \times d_y - F_{Dy} \times d_z \label{EQ_dragMomentsL}\\
		M_{Dy} = F_{Dx} \times d_z - F_{Dz} \times d_x \label{EQ_dragMomentsM}\\
		M_{Dz} = F_{Dy} \times d_x - F_{Dx} \times d_y \label{EQ_dragMomentsN}
		\end{eqnarray}
		
		\begin{figure}[H]
			\centering
			\includegraphics[height = 8cm]{Images/Literature/drag.jpg}     
			\caption{Typical moments created by drag forces}
			\label{IM_dragForces}
		\end{figure} 
				
		\subsubsection{Airflow Characteristics}
		
		In the preceding section on flight theory, the importance of air density, pressure and the creation of rotor wake boundary are discussed. The negative effects of disrupting airflow as well as the need for controlling this disturbance has been well documented in literature \cite{NearWall, Lee2012, Hoffmann}.
		
		Using Figures \ref{IM_MomentumTheoryAirFlow} and \ref{IM_MomentumTheoryHover} as references Airflow can be seen as the stream of air from $v_0$ to $v_\infty$\, through $v_i$. Equation \ref{EQ_ThrustMass} states that thrust is directly proportional to the relationship between these velocities and any deviation in these velocities will vary the thrust of the rotor in question. $v_0$ is only zero when the craft is in a state of pure hover, completely stationary, and there is no wind. Increasing the speed of the craft will increase the $v_0$ component creating a variation in the overall thrust, the same can be said for any condition that contains a tangible wind factor. 
		
		As investigated by \cite{Hoffmann} mechanical intrusions will have an effect on the far wake velocity, thus also effecting the generated thrust. In the design of STARMAC by Hoffmann et al \cite{Hoffmann} the frame was designed to be very configurable so that the effects of the mechanical design could be quantified. Originally the rotors were shrouded and quite close to the centre of gravity of the craft. The shrouds were a distance of 5\% rotor radius and when removed the yaw tracking improved from $\pm$10\textdegree to $\pm$3\textdegree. When not included in the dynamic model the obstruction in the air stream will cause lower and less stable values of thrust, affecting the accuracy of the model. 
		
		When the rotors were located close to the centre of gravity they obtained some attitude disturbances that were eliminated by moving the rotors further away. It was also observed that any object that lies close to the rotor tip, created intense arbitrary disturbances and should be avoided \cite{Hoffmann}.
		
		\cite{NearWall} attempts to model some of the disturbances for a single rotor craft hovering near wall, but as stated by \cite{Lee2012} it is not viable to accurately quantify these disturbances, however their presence must not be neglected. As Figure \ref{IM_NearWallF} shows, these can be modelled as a disturbance to the input force and moments.
		
		\begin{figure}[H]
			\centering
			\includegraphics[height = 7cm]{Images/Literature/near_wall}     
			\caption{Velocity components though the rotor for, no wall (left) and near wall (right) conditions (Taken from \cite{NearWall})}
			\label{IM_NearWallF}
		\end{figure}
		
		As demonstrated by \cite{NearWall}, there is also an induced moment acting on the rotor as the rotor approaches the wall. Figure \ref{IM_NearWall} is an image generated by \cite{NearWall}, it demonstrates the change in airflow on a rotor close to a wall region.
		
		\begin{figure}[H]
			\centering
			\includegraphics[height = 5cm]{Images/Literature/NearWall}     
			\caption{Velocity components though the rotor for, no wall (left) and near wall (right) conditions (Taken from \cite{NearWall})}
			\label{IM_NearWall}
		\end{figure}
		
		In \cite{NearWall}, Robinson et al used the script $c$ as their unit of measure for distance to the wall, $c$ is chord length of the airfoil. The graph shown in Figure \ref{IM_NearWallGraph} shows how the moment felt by the craft varies with the distance to the wall. The direction of the moment will be perpendicular to the wall. As above, this disturbance can also be modelled as a variation to the input moments to the system.
		
		\begin{figure}[H]
			\centering
			\includegraphics[height = 5cm]{Images/Literature/NearWallGraph}     
			\caption{Graph showing relationship between distance from the wall and moment felt be the craft (Taken from \cite{NearWall})}
			\label{IM_NearWallGraph}
		\end{figure}
		
		In \cite{NearWall} they conclude their paper by describing a proposed method of control which will be investigated further in this text. This dynamic flight model requires sensing of specific craft variables, typical sensing methods and requirements are discussed in the proceeding section. 
	
	\subsection{Instrumentation}
	The need for a well instrumented craft is intuitive and well documented in literature \cite{Moller2015, Hoffmann}. With modern advancements in sensing technology there is now a number of methods to obtain the required information. However, some of these are very costly and require specific operational environments. For this purpose, only the inertial measurement unit (IMU) and the global positioning system (GPS) are discussed here.
	
		\subsubsection{Inertial Measurement Unit}
		Traditionaly an IMU will consist of an accelerometer, gyroscope and possibly a magnetometer. The accelerometer has the ability to measure accelerations in the body axis. It should be noted that an accelerometer in rest, sitting along the inertial axis will provide a reading of $[0\  0\  g]^T$. The gyroscope measures the rotational accelerations of the craft, thus measuring 6 degrees of freedom in total. Unfortunately gyroscopes suffer from sensor drift and needs compensation. The accelerometer's gravity offset can be used to align with the vertical axis, however a magnetometer is required to account for drift in the horizontal plane. Various filtering and fusion techniques are used to combine these readings, the most popular of which  is a Kalman Filter \cite{Moller2015, Hoffmann}.
		
		\subsubsection{Global Positioning System}
		The most common method of measuring a drone's position is with the use of a GPS. The GPS readings can also be used to create an estimate of the craft's linear velocity in the inertial frame. When combined, multiple GPS unit's can be used to obtain higher precision. A major downside of a GPS is the dependence on a satellite link, this dependence severely limits the operational environments for this sensor. Other methods of localisation include using known maps and proximity sensing to provide the rotorcraft with position estimates. As discussed, these techniques are out of scope for this project and will not be considered. This project will assume an earth position and velocity estimate are present, subject to filtered noise provided through band limited white noise \cite{Moller2015}
		
		\subsubsection{Measurement Noise}
		From the discussion it is evident that sensors will not be without error. The noise associated with the measurements will be different for both varying sensor types and manufacturers and will depend on the chosen sensors. The characterisation of the noise for this project is discussed in the system identification chapter. The environment in which these sensors operate also has a significant effect on operation. Moller characterises this measurement noise as a random band limited white noise block. and adds a low pass filter to the GPS measurements \cite{Moller2015}.

\section{Collision Protection Techniques}
	\todo[inline]{Need to decide if I should keep this in,l and in what format}
	\cite{Klaptocz2013, Collision, Klaptocz2012, Briod2012, Daler2013, Klaptocz2010}
	\subsection{Impact Resistance}
	%Refer to impulse equations, Newtons's laws
	%look at car designs
	Impact resistance is a technique used by a variety of fields in the world today. Included in this is something as simple as the shocks or suspension in a car, they are designed to allow the automotive to withstand sudden impacts. Generally these techniques use a component that has some tangible spring constant.
	
	
	\subsection{Rolling Cage}
	\todo[inline]{Although a lot can be learnt from this design in terms of collision resistance, for the proposed end use case the design will not be pursued. The rolling cage design}
	
\section{Review of Existing Flight Control Strategies}\label{SECT_ControlReview}
The object of this section is gain a better of understanding of successful controller designs. Moller's control structure was used as a basis for this section \cite{Moller2015}. A simplified block diagram is shown in Figure \ref{IM_ControlOverviewSimplified}. The open source software framework, ArduPilot is a well developed code base. The quadrotor code was also reviewed to establish an understanding of implementing these 

 Along with an analysis of an existing open-source framework. For a fully autonomous vehicle, the controller would need to include multiple control loops.

This section begins with
\begin{figure}[H]
	\centering
	\includegraphics[height = 6cm]{Images/Literature/TiltRotor}     
	\caption{Hager's design for a telescopic tilt rotor system \cite{Heli}}
	\label{IM_ControlOverviewSimplified}
\end{figure}

The high level control structure can be decomposed into 
		
	\subsection{Altitude Control System}
	
	\subsection{Heading Control System}
	
	\subsection{Horizontal Control System}
	
	\subsection{Disturbance Based Observer Controller}\label{SectionDisturbanceObserver}
	Discuss and explain generic disturbance based observer stuff. In the control section discuss specific implementation. Mention standard equations here. 
	
	
\section{Collision Avoidance Techniques}

	\subsection{Sensing Techniques and Requirements}
	
		\subsubsection{Modelling}
		
	\subsection{Potential Field Method}
	
	
	
	
	
	
	
	
	
	
	
	
	
	
	
			
\chapter{Platform Design}
Obtaining successful flight results with an aerial vehicle will need both precise control algorithms and a robust platform. This chapter contains the technical details of how the final platform decisions was made. The final design will be compared and validated against the proposed outcomes stated above. 

\section{Factors Affecting Design Decisions}
\todo[inline]{not sure about the name of this section}
Before a proper analysis can be done, certain criteria need to be outlined. These include aspects of the project such as the required flight time and manoeuvring decisions. These factors have been drawn up from the proposed research questions and end use case.

\subsection{Physical Restrictions and Requirements}
One of the major components \projectName will have to overcome is to navigate these unknown areas and not only survive collisions, but also reject the disturbances introduced by being close to the walls. Since mine shafts are predominantly long and narrow, the same approach will be looked at for the design of the drone. Hence the platform will be made to be long and narrow. The unique disturbances found in a mine also help define the end configuration as it needs to be able to withstand these disturbances.
Since the drone will be required to navigate very confined spaces, the smaller the drone the better. The minimum size of the drone is limited by the need for adequate flight time as well as payload capacities.

\subsection{Manoeuvring Decisions}
The manoeuvring decisions are dependant on the type of environment and type of missions required by the platform. These decisions influence the final design of the craft.
The end use case for \projectName will include mapping of unknown areas. To complete this, it will simplify the procedure if the drone keeps it's orientation during flight. Due to the nature of the environment, fast speeds will not be used regularly. Therefore \projectName will be designed to have slow, steady and controlled movements. More is discuss in the flight strategy section.

\subsection{Disturbances}
As described above the discussions with CSIR lead to the decision to attempt to use this drone in a mine. Apart from the difficulty of navigating and manoeuvring through an unmapped area there are other disturbances introduced into the system. 
Due to the nature of the tunnels, wind gusts are created that funnel through these passageways. They can reach speeds of up to \todo{Insert Gusts Speed and reference}. Due to the direct relationship between thrust and initial velocity as shown in \eqref{EQ_ThrustMass} any change in wind speed through the rotors will unbalance them.
The effect of coming close to a wall of floor has been discussed in the literature study. The near wall effect will also create an imbalance that creates a moment towards the wall. 
Since the areas will be unknown, collisions and bumps are extremely likely, if not inevitable. The drone must be able to withstand a collision and maintain it's orientation as best as possible.

\subsection{Thrust Overhead}
The total overhead of a rotorcraft is a percentage above the thrust required for hover. This value determines a craft's ability to manoeuvre, with a higher value giving it more freedom and a greater ability to resist disturbances. With these benefits the system does become very sensitive and more difficult to control and stabilise. Since the craft will be in confined narrow passages, the craft does not need to be fast moving. Rather a "slow and steady" pace will be approached. The craft does however need enough power to counteract the disturbances described above. These considerations lead to a value of $150\%$, with $100\%$ being enough thrust to hover. \todo[inline]{Back this up with references from other designs, spoken about later consider removing from here. Maybe reference botom section?}

\subsection{Flight Time}
Flight time is dependant on efficiency/power requirements of the system as well as the capacity of the on-board power source. This is where the designer runs into a catch 22. By adding a larger power source the weight is increased and therefore the platform requires more power to keep itself aloft. Weight is a determining factor for any aerial system and influences flight time, for this reason the weight of every subsystem must be optimised. To ensure the craft can complete a mission it will need a decent flight time. Initial conversations have set 30 minutes as the bottom limit. The original platform might not be able to reach this goal, but once the platform is performing adequately adjustments can be made to the system to optimise weight and power consumption. 

\section{Final Concept Choice}
The traditional configurations of drones are incapable of handling disturbances introduced by being in a confined space and will continue to struggled to overcome this problem\todo{Maybe a bit harsh, but need to validate why we're not using a standard design}. For this reason a few unique designs were considered and a lot subsequently rejected. Although some rejected ideas did help define the final configuration.

\subsection{Rejected Designs}
\subsubsection{Gimball}
The Gimball design as shown in Figure \ref{IM_Gimball} has been discussed thoroughly in the literature. It is definitely the most sophisticated close quarter drone in the world today. It's rolling cage and gimbal design reduces collisions to hardly effect the drone's flight paths at all. Unfortunately it's Co-axial rotors limit the payload capabilities and reduce efficiency of the craft which inhibits the total flight time. The rolling cage also prohibits successful mapping as the orientation of the craft is constantly changing and any sensors inside the cage will be obscured by the rolling protection.

\begin{figure}[H]
\centering
\includegraphics[height = 6cm]{Images/Design/Gimball}
\caption{Flyability's Gimball (Picture taken from http://www.flyability.com/product/)}
\label{IM_Gimball}
\end{figure}

The Gimball is a great tool and will aid a lot of industries, but it was not seen as a viable method to conduct mapping or carry any sort of payload for an extended period of time.

\subsubsection{Tandem Tilt Rotors}
After much thought into the problem a unique design was considered. Figure \ref{IM_TiltTandem} is the initial realisation of that concept. The idea was to capitalize on a tandem's narrow design while giving it more freedom by allowing the rotors to tilt. After the concept had been drawn up it was noted that adding tilt abilities to standard configuration has been looked at and used successfully before \cite{TandemTiltRotor, TripleTiltRotor}.

\begin{figure}[H]
\centering
\includegraphics[height = 6cm]{Images/Design/Tandem}
\caption{Rendering of initial concept of the Tandem Tilt rotor Rotorcraft}
\label{IM_TiltTandem}
\end{figure}

The big issue with this design would be controlling the platform. Designing a tilt rotor without variable tilt rotors would not produce enough roll control. The tilt mechanism will help with this, but affect too many other efficiencies while doing it. The tandem will also be more susceptible to disturbances than a four rotor approach.

\subsection{Chosen Concepts}
\todo[inline]{For the comparison, figures were obtained while the following was assumed: Thrust to RPM linear. RPM to current linear. Make sure they are correct}

After much thought two concepts were selected as final candidates. This next section walks through some of the important factors considered and ultimately, why certain decisions were made. 
On the left of Figure \ref{IM_UnlikeSizes} represents the \textit{"Unlike Size Quad"} and the right of Figure \ref{IM_UnlikeSizes} is the \textit{"Overlapping Quad"}. 

\begin{figure}[H]
\centering
\includegraphics[height = 6cm]{Images/Design/UnlikeSizes}
\includegraphics[height = 6cm]{Images/Design/Overlapping}
\caption{a)Rendering of initial concept of the unlike rotor size quadcopter and b) Malloy Aeronautics Hoverbike Concept (Picture taken \cite{MAHover}).}
\label{IM_UnlikeSizes}
\end{figure}


\subsubsection{Concept 1 - The Overlapping Quad}
The overlapping quad is a concept pursued by Malloy Aeronautics \cite{MAHover}. They used the design in an initial concept of their hover bike. The design uses an X-formation for it's rotors, except the rotors are brought in to limit the width of the craft to the point that they overlap, as shown in Figure \ref{IM_UnlikeSizes}. Each overlapping pair will have both spin directions, this feature is shown in Figure \ref{IM_OverlappingPair}.

\begin{figure}[H]
\centering
\includegraphics[height = 6cm]{Images/Design/OverlappingVD}
\caption{Overlapping concept, visual representation of rotor pairs. Image modified from \cite{MAHover}}
\label{IM_OverlappingPair}
\end{figure}

If this configuration is selected, there are still other design questions that need to be answered. Overlapping rotors introduce an inefficiency into the system. Johnson in \cite{HeliTheory}, says there is much debate in how the efficiency is calculated. He states two of his preferred methods, the method chosen has a better approximation for small overlaps and is shown in \eqref{EQ_OverlapEfficiency} \cite{HeliTheory}. $\Delta P$ is the interference power (considered here as a fraction of total power) and $m$ is the overlap fraction and is calculated in \eqref{EQ_Overlap} \cite{HeliTheory}.

\begin{equation}
\label{EQ_OverlapEfficiency}
\frac{\Delta P}{P} = (\frac{2}{2-m})^{1/2} - 1
\end{equation}

\begin{equation}
\label{EQ_Overlap}
m = \frac{2}{\pi} \Bigg[ \cos^{-1}\frac{l}{2R} - \dfrac{l}{2R}\sqrt{1 - {\dfrac{l}{2R}}^2} \Bigg]
\end{equation}

These quantities assume a small vertical separation so that the inflow rates of both rotors can be considered the same. To calculate the overlap function, the rotor radius $R$ is needed as well as the separation distance $l$. Figure \ref{IM_SeperationGraph}, illustrates how the separation affects both the  overlap function as well as the total power as a percentage.


\begin{figure}[]
\centering
\includegraphics[height = 6cm]{Images/Design/SeperationGraph}
\caption{Graph representing the effects of separation distance in an overlapping quad}
\label{IM_SeperationGraph}
\end{figure}


The length of the craft needs to also be decided, this aspect only becomes critical once payload requirements are factored in. As stated above, this design was an initial concept for a hoverbike with the intended payload capacity of a human. So a big positive to this design is the power to size ratio. This benefit can be utilised through handling larger payloads, or a larger battery pack increasing total flight time.

\subsubsection{Concept 2 - The Unlike Size Quad}
The unlike size quad is an original design that uses the standard cross formation, except it has two pairs of different size rotors. This means that the counter rotating pairs will be set up as shown in Figure \ref{IM_UnlikeSizePair}, with each rotation direction having one big and one small rotor. 

To maintain a common DL in the system the thrust requirement will be lower on the smaller blades and larger on the bigger blade. The smaller the side rotors get, the higher the thrust requirements of the larger rotors become, this limits the rotor size ratio. Initial calculations, factoring in thrust overhead, overall size of the craft and minimum thrust allowed to rotors, set the ratio between $65\% - 80\%$.  When approaching the lower bound, the thrust requirement for hover alone leaves very little room for manoeuvring or disturbance rejection. The upper bound reaches a point where the size difference is so negligible the design's narrow intent is lost.

\begin{figure}[H]
\centering
\includegraphics[height = 6cm]{Images/Design/UnlikeSizeRotorPair}
\caption{Unlike Size Quad visual representation of rotor pairs}
\label{IM_UnlikeSizePair}
\end{figure}

The second important choice is how far to put the rotors away from the centre. As the craft gains translational speed, the air doesn't come in directly from the top any more as shown in \ref{IM_MomentumTheoryAirFlow}. Instead it now starts to come in at an angle. As the craft then manoeuvres and changes it's orientation, the air starts coming in at more extreme angles. If the electronics housing has a lot of height it can interfere with the wake boundary and create an inefficiency. This inefficiency is also based on how far the rotors are from the housing. A similar concept is described in Section \ref{SSSECT_NearGroundEffect} Therefore, before this decision can be made the limits of how small the centre electronics housing needs to be decided. 

Through discussions and observations of current systems a minimum limit of $75mm \times 75mm$ was set. To avoid overlap, the distance from the centre of the big rotor and the centre of the craft will be at the very least $R$. Including space for the housing increases this.

The left and right placement is where the originality comes in. Since the craft needs to remain narrow, the side rotors can be brought in as well as shrunk. This would require pushing the two larger rotors slightly more out. Bringing the side rotors in closer to the middle housing will reduce efficiency, so a compromise must be chosen between width and efficiency. Since the side rotors don't contribute as much to the system as their bigger counterparts they have the option of a lower separation distance. The lower separation distance can also be justified by the lower need and use of roll moments and side translations.

Lastly the rotor pitch angle needs to be selected. The major contributing factor of any rotor (besides it's size) is it's pitching angle. As the pitch angle changes, so does the lift to drag ratio. Generally a craft would always want a higher lift to drag ratio, in this case however the side rotors might have a lower lift to drag ratio so they can help counter the torque applied by the larger blades more effectively.

\subsection{Concept Comparison}
After being introduced to the two concepts, they will now be compared based on certain factors. This comparison includes hover efficiency, thrust and electrical power. Lastly size and manoeuvrability are grouped together since they inherently affect each other. The final decision will need to be made with certain assumptions in mind, that must be validated through testing further down the line. These assumptions as well as the method of comparison are described below. 

\subsubsection{Assumptions and Method}
Since both designs use 4 rotors they can be compared relative to a well known design, the standard quadrotor. For each configuration certain parameters need to be decided before a comparison can be done. Using formulas and plots in Microsoft Excel\texttrademark, both designs could be visualised rudimentary as shown in Figure \ref{IM_Excel}.

\begin{figure}[H]
\centering
\includegraphics[width = 15cm]{Images/Design/Excel}
\caption{Rudimentary Visualisation of the two concepts using Microsoft Excel\texttrademark}
\label{IM_Excel}
\end{figure}

If hover is a thrust of $100\%$, the overhead was set at $50\%$ \todo{Try and find papers stating other values I want to validate this decision} of that, to a total of $150\%$. Minimum thrust per rotor was set at $20\%$. The mass in question includes provision for a sensor pack of undecided mass. 

The two designs had different decisions that needed to be made. After a bit of trial and error the unlike size quad had it's small rotors set at $75\%$ of the larger ones, this decision is explained further in the text. Just to give a quantifiable reference, R was set at $254mm$. With that assumption the unlike size quad moves it's side rotors in to a distance of $300mm$ and the larger rotors were moved to $508mm = 2R$ away. The overlapping quad set a separation distance of $350mm$ which created an overlap factor of $m = 19.82\%$. 

These values were used to do the following analysis.


\subsubsection{Rotor Area and Disk Loading}
If a rotor size of $R$ is assumed for the rotors\footnote{The big rotors in the case of the unlike size quad} then the total area for a standard quad will be $A_{std} = 4 \pi R^2$. Based on simple geometry, the reduction in radius of the two smaller rotors leads of the unlike size quad leads to a decrease in area. Using $75\%$ this is calculated to $78.13\% of A$. 

As described in \eqref{EQ_Overlap}, the overlapping quad introduces a reduction in total disk area, with an overlap factor of $19.82\%$ leaves a total area of $80.18\%$. Since the thrust requirements are assumed the same, the disk loading ratios will be exactly the same as the area ratios.

\subsubsection{Thrust and Power Considerations}
The $75\%$ decision was based on observation of thrust ratios and comparing them to the minimum and maximum values stipulated earlier. This decision must also be influenced by available rotor sizes and pairings. The final value graphs are shown in Figure \ref{IM_ExcelGraphs}. The thrust percentage per rotor is shown, the points marked are at minimum, hover and maximum. Equation \eqref{EQ_ThrustMass} states that thrust is proportional to area. Therefore the reduction in rotor area will cause a directly related reduction in thrust. The total thrust available to the unlike size quad is $\approx 78\%$ of the thrust available to the standard design. This reduced value also comes at a weight reduction. The overlapping quad has an equal total rotor area but an inefficiency is introduced by the overlap as according to \eqref{EQ_OverlapEfficiency}. Therefore the overlapping quad has $94.64\%$ of the total thrust, without the weight reduction.

\begin{figure}[H]
\centering
\includegraphics[height = 10cm]{Images/Design/ExcelGraphs}
\caption{Graphical representations of the thrust ratios for the unlike size quad}
\label{IM_ExcelGraphs}
\end{figure}

As for the electrical power, the values were calculated according to how much energy would be needed to obtain the same thrust as the standard design. The inefficiency introduced by the overlap relates to a reduction in thrust of $\Delta T_{overlap} = 5.36\%$, therefore $\Delta P_{overlap} = 14.21\%$ is needed to overcome this loss, based on \eqref{EQ_ElectricalPowerThrust}. 

For the unlike size quad, the reduction in rotor size leads to a substantial loss in aerodynamic power, even with a small reduction in inertia. To regain that power, the rotors need to be pushed harder, this leads to an increase in electrical power. A value of $\Delta P_{unlike size} = 18.5\%$ can be calculated under certain assumptions.

\subsubsection{Size and Manoeuvrability}
The size was calculated as though the drone made a rectangular box and with the rough values above, Table \ref{TAB_SizeComparison} puts those values in a tabular format.

\begin{table}[H]
	\centering
	\begin{tabular}{l | c | c }
		Concept & Length ($mm$) & Breadth ($mm$) \\
		\hline\hline
		Unlike Size	   	& 1524 & 981 \\
		Overlapping    & \boldmath$1308$ & \boldmath$858$ \\
		Standard		& 1524 & 1524\\
	\end{tabular}
	\label{TAB_SizeComparison}
	\caption{Table representing the size comparison of the two concepts}
\end{table}

As shown Both crafts are similar in size, with the overlapping design being slightly shorter as well as more narrow. 

This narrowness comes at a cost. Moments ($M$) are created by a difference in forces($\Delta$) and is multiplied by the distance between the forces ($l$) $M = \Delta Fl $. In the case of the overlapping rotors, the distance between the two different thrust vectors is not as tangible as the unlike size design. This gives the unlike size quad the advantage as bigger moments can be created\todo{Need to make this quantifiable}, allowing for more advanced manoeuvrability and disturbance rejection. However in the case of the near wall effect, the disturbance is created by effecting only one rotor . In the case of the overlapping rotors, the disturbance might be slightly diminished as both rotors will feel the effect. This can be verified later by comparing gathered data to literature. \todo{Would be cool to come back and do this later and reference it here}


\section{Platform Conclusions}

The quantifiable values are culminated below in Table \ref{TAB_ConceptComparison}. The winner of each parameter is written in bold.

\begin{table}[H]
	\centering
	\begin{tabular}{l | c | c | c | c | c }
		Concept & Disk Loading & Total Thrust & Electrical Power & Length ($mm$)& Width ($mm$) \\
		\hline\hline
		Unlike Size	  & \boldmath$78.13\%$  & $78.13\%$ 	& $118.57\%$	& $1524$ & $981$ \\
		Overlapping    & $80.18\%$ & \boldmath$94.64\%$  & \boldmath$114.21\%$	& \boldmath$1308$ & \boldmath$858$ \\
		\hline\hline
		Standard		& $100\%$ 	& $100\%$  	  & $100\%$			& $1524$ & $1524$\\
	\end{tabular}
	\label{TAB_ConceptComparison}
	\caption{Table representing the end comparison of the two concepts}
\end{table}

Now to make the final decision all the factors need to be included. What is known now is that the overlapping quadrotor is superior in almost every quantifiable way. The manoeuvrability aspect is not critical to this application as the use case will involve extremely steady flight. The disturbance rejection of both crafts will be good with 4 rotors. 

Malloy Aeronautics sell what is called a \textit{Drone-3 Kit}, it includes various accessories to help developers use the platform. The novelty of the unlike size rotor quad can now be seen as a negative since it requires a custom frame and shroud.

Based on the above analysis and conclusions the overlapping quadrotor design will be used as the design going forward.

\section{Drone 3}
\subsection{Assembly}




\chapter{Mathematical Modelling and System Identification}
	
\section{Dynamic Flight Model}\label{SSECT_DynamicFLightModel}
\todo{Put in how the flight model in simulink works etc}
The dynamic flight model of the craft must cater for all six of the degrees of freedom the craft experiences. The dynamics of this system is modelled as of three rotational and three translational degrees of freedom \cite{Moller2015}. To continue deriving the equations of motion, the following assumptions are made: the aircraft is a rigid body, the aircraft has constant mass, $I_{xy}, I_{xz}, I_{yz}$ are all negligibly small.   

The Newton-Euler method of defining the accelerations uses the inertial frame to define the linear accelerations, and the body frame to define the rotational accelerations. Using Newton's first law and the rotation matrix described in \eqref{EQ_RotationMatrix}, the expression for the linear acceleration can be developed and is shown in \eqref{EQ_EulerNewtonInertialMatrix}.

\begin{equation}
\begin{bmatrix} \ddot{N}\\ \ddot{E}\\ \ddot{D} \end{bmatrix} = \begin{bmatrix} 0\\ 0\\ g \end{bmatrix} + \textbf{R} \begin{bmatrix} 0\\ 0\\ \frac{\textbf{T}}{m} \end{bmatrix}
\label{EQ_EulerNewtonInertialMatrix}
\end{equation}

The rotational accelerations of the craft can be similarly described using the moments and the simplified inertia tensor. These rotational rates are described in \eqref{EQ_EulerNewtonRotationMatrix}

\begin{equation}
\begin{bmatrix}\ddot{\phi}\\ \ddot{\theta} \\ \ddot{\psi} \end{bmatrix} = \frac{1}{\boldmath{I}} \times \begin{bmatrix} N  \\ M  \\ L \end{bmatrix} = \begin{bmatrix} I_{xx} N  \\ I_{yy} M  \\ I_{zz} L \end{bmatrix}
\label{EQ_EulerNewtonRotationMatrix}
\end{equation}
	
\section{System Identification}
In order to correctly model the system, a thorough system identification needs to be completed. This entails real world measurements of the chosen platform. The methods and results from these experiments are covered in this section.

	\subsection{Mass and Inertia}
	Using a calibrated scale the mass of the rotorcraft measured at 3.352Kg. To calculate moments of inertia, the Bifilar Pendulum method was used. The method is thoroughly described in literature and involves tying the drone from the ceiling allowing it rotate around one axis. Since it is desired to measure the inertias along three axes, three separate test set ups were required. Images of the test set up for a single axis is shown in \ref{IM_BifilarPendulum}. 
	
	\begin{figure}[H]
		\centering
		\includegraphics[height = 7.5cm]{../References/Diagrams/ControlDiagram.jpg}
		\caption{Bifilar Pendulum for Measurement}
		\label{IM_BifilarPendulum}
	\end{figure}
	
	Each axis was measured 10 times and the values were averaged out to obtain the final values represented in Table \ref{tab:MomentOfInertia}. To give a representation of measurement variance, the standard deviation is provided along side.
	
	\begin{table}[!]
		\centering
		\begin{tabular}{l | c | c | c |}
			Parameter & Averaged Measured Value & Standard Deviation\\
			\hline\hline
			$I_{xx}$ & 0.025027578 & 0.001063842\\
			$I_{yy}$ & 0.169260024 & 0.000142928\\
			$I_{zz}$ & 0.170196714 & 0.000527406\\
		\end{tabular}
		\label{tab:MomentOfInertia}
		\caption{Measured Moments of Inertia}
	\end{table}
	
	\subsection{Thrust and Moment Profiles}
	In order to correctly valid the thrust characteristics of the drone, each motor rotor pair needed to be evaluated. Each pair was marked and coupled to a load cell. The ESCs were configured to send varying PWMs to the motors. The commands sent to the ESCs and the measured thrust values are plotted together in Figure \ref{IM_ThrustProfiles}.
	
	\begin{figure}[H]
		\centering
		\includegraphics[height = 7.5cm]{../Design/Mechanical/ThrustProfiles/thrustprofiles.jpg}
		\caption{Bifilar Pendulum for Measurement}
		\label{IM_ThrustProfiles}
	\end{figure}
	
	\begin{table}[!]
		\centering
		\begin{tabular}{l | c | c | c |}
			Pair & Max Thrust & Min Thrust\\
			\hline\hline
			$1$ & 18.8558 & 0.7852\\
			$2$ & 19.1489 & 1.1889\\
			$3$ & 19.1434 & 0.9511\\
			$4$ & 19.3369 & 0.5087\\
		\end{tabular}
		\caption{Measured Moments of Inertia}
		\label{tab:ThrustProfiles}
	\end{table}
	
	\subsection{Drag Coefficients}
	\todo{How did you choose these?}
	
	\subsection{Sensor Constants}
	\todo{How did you choose these?}
		\subsubsection{Sensor Offsets}
		\todo{How did you choose these?}
		\subsubsection{Measurement Noise}
		\todo{How did you choose these?}
	\section{Instrumentation}
	\todo{Here is where you put how you created the instrumentation models and noise, based on what you learnt in the lit review}
	
	\section{Disturbances}
	\todo{Here is where you put how you created the disturbance models, based on what you learnt in the lit review}
	\subsection{Wind Model}
	\subsection{Drag Model}	
			
	\section{Discussion}
		
\chapter{Controller Design}
This chapter follows the stages of designing a controller for a quadrotor and begins by discussing the overall strategy. An analysis of existing rotorcraft control systems has been done in Section \ref{SECT_ControlReview}. The system must be able to follow waypoint commands, containing a position reference in the North, East and Down frame. As well as align with a desired heading, this reference will be in the form of a Euler angle. The detailed design goals are discussed before the controller design is begun.

\section{Design Goals}


\section{Flight Control Strategy}
Figure \ref{IM_ControlStrategy} represents the high level control strategy for this project. This discussion will break the system into three main components: an altitude controller, a horizontal flight controller and a heading controller.	

The altitude controller starts by getting a position reference in the earth frame. This reference is fed through a Proportional Integral (PI) controller to create a desired climb rate. The climb rate reference is converted into the body frame and is then controlled using a Proportional (P) controller to produce a body acceleration reference. The heave controller can only produce a force perpendicular to the rotor, thus the Z-Axis acceleration component is taken and fed through an additional PI heave controller. The output of this inner heave controller would be the virtual actuator $\delta_Z$.

The horizontal controller gets both a North and an East position reference and uses PI controllers to create velocity setpoints in the earth frame. These setpoints are converted to the body frame where the linear velocity P controller works and outputs acceleration references in the body frame. Using the body acceleration references, roll and pitch rate setpoints can be created using a tilt angle controller. These rate set points are fed into the inner rate loop lead lag controllers which output the virtual actuators $\delta_\phi$ and $\delta_\theta$.

Lastly the heading controller is discussed. Correct design of the tilt angle controller will decouple the vehicle's horizontal controller from a dependency on the heading of the craft. This allows seamless implementation of a heading controller. The heading controller is broken down into two parts: a yaw angle and a yaw rate controller. The angle loop uses a PI control architecture while rate loop utilises a simple P controller. The output of the yaw rate controller is the virtual actuator $\delta_\psi$.

\begin{figure}[H]
 	\centering
 	\includegraphics[height = 10cm]{../References/Diagrams/HighLevel.jpg}
 	\caption{High Level Control Strategy}
 	\label{IM_ControlStrategy}
\end{figure}

Each controller utilises the same rotor system to produce their outputs. This dependency on a single generation system allows for interference between the controllers. To ensure that no loop can saturate another, each system is given a percentage of headroom in which it can work, as shown in Table \ref{tab:HeadRoomPercentages}. The controller design must ensure that the thrust commanded during a step response is within those limits.

\begin{table}[H]
 	\centering
 	\begin{tabular}{l | c | c |}
 		Controller 				& Percentage & Allowed Thrust Per Rotor\\
 		\hline\hline
 		Altitude 	   			& 64\% & 12.16N	\\
 		North  		    		& 8\%  & 1.52N	\\
 		East					& 12\% & 2.28N	\\
 		Heading 		    	& 6\%  & 1.14N	\\
 		Safety Factor 			& 10\% & 1.9N   \\
 	\end{tabular}
 	\caption{Thrust Headroom Controller Percentages}
 	\label{tab:HeadRoomPercentages}
\end{table}
	 
\section{Altitude Controller}
This section discusses the design and implementation of the altitude controller which is responsible for controlling the desired height of the craft. The craft is required to fly in confined spaces and must be able to track a setpoint with zero steady state error and negligible overshoot. In order for the altitude system to handle disturbances, an integrator term will be required. The system must be able to respond quickly to commands, however it must not exceed it's thrust utilisation percentage. To do this the system will require an upper limit which should not be reached. A lower limit is then introduced so the craft does not descend too quickly.

The overall altitude controller is structured as a set of cascaded control loops, with the most inner loop controlling the aircraft's acceleration and the most outer loop controlling the desired altitude in the earth frame. The system must be able to respond to disturbances quickly. Therefore, the inner heave loop utilises a PI controller to follow a desired vertical acceleration reference. The climb rate P controller is responsible for generating these acceleration references and is fed a desired linear vertical velocity reference by the altitude hold P controller. In order to reject measurement errors in the inner loops, the altitude hold controller makes use of a limited integrator that does effect the bandwidth of the system. Before the controller's can be designed, an analysis of the system's heave dynamics must first be performed.
	 
	 \subsection{Heave Dynamics}
	 Using Newton mechanics at near hover conditions for the aircraft, the heave dynamics can be derived and are shown in \eqref{EQ_HeaveNewton}. Where $\dot{W}$ is the current acceleration of the craft in the Z-Axis and $m$ is the vehicles's mass. $Z$ is defined as the current instantaneous force being produced by the rotors.
	 
	 \begin{equation}
	 \label{EQ_HeaveNewton}
	 \dot{W} = \dfrac{Z}{m}
	 \end{equation}
	 
	 The state variable of the system is chosen as $Z$ with the output of this plant being $\dot{W}$. Using the transfer function for motor-rotor lag dynamics seen in \eqref{EQ_MotorDelay} and the dynamics seen in \eqref{EQ_HeaveNewton}, the state space equation for the system can be derived and is shown in \eqref{EQ_HeaveStateSpace1} and \eqref{EQ_HeaveStateSpace2}. 
	 
	 \begin{eqnarray}
	 [\dot{Z}] &=& - [\dfrac{1}{\tau}] \ [Z] + [\dfrac{1}{\tau}] [\delta_Z]\label{EQ_HeaveStateSpace1}\\\label{EQ_HeaveStateSpace11}
	 [\dot{W}] &=& - [\dfrac{1}{m}] \ [Z]\label{EQ_HeaveStateSpace2}
	 \end{eqnarray}
	 
	 Subsequently the transfer function can be calculated and the result is shown in \eqref{EQ_HeaveTF}. The negative gain of the transfer function must be noted and is caused by the direction of the defined axes, with the rotors producing a negative Z-Axis force.
	 
	 \begin{equation}
	 G(s)_{heave} = \frac{\frac{-1}{m \times \tau}}{s + \frac{1}{\tau}}\label{EQ_HeaveTF}
	 \end{equation}
	 
	 The rotor motor lag produces the pole at $\dfrac{-1}{\tau} = -8$ and indicates the maximum response capabilities and timing constant of the rotor system.
	 
	 \subsection{Heave Controller}
	 The heave controller is responsible for commanding the $\delta_Z$ virtual actuator to achieve a desired Z-Axis acceleration in the body frame. The heave controller is the fastest controller in the altitude system and should utilise as much bandwidth as the rotor-motor system allows. The altitude controller wishes to reject disturbances quickly and thus a PI architecture was initially chosen as shown in Figure \ref{IM_HeaveController}. The system should be stable and exhibit a phase margin of at least $70$\textdegree.
	 
	 The integrator is used to reject disturbances, while the most left limiter shown in Figure \ref{IM_HeaveController} was added to stop integrator wind up and is not considered during the linear controller design. The proportional gain is used to move the closed loop poles and achieve the desired bandwidth. The dynamic response of the system can be investigated using the root locus and bode plots shown in Figure \ref{IM_HeaveControlRoot} and \ref{IM_HeaveControlBode}. 
	 
	 \begin{figure}[H]
	 	\centering
	 	\includegraphics[height = 3.5cm]{../References/Diagrams/HeaveController.jpg}
	 	\caption{Heave Controller -  Control Diagram}
	 	\label{IM_HeaveController}
	 \end{figure}
	 
	 Figure \ref{IM_HeaveControlRoot} shows the root locus of the system with the PI controller included. The controller introduces a new open loop pole at the origin. To maintain a first order response, the zero is placed close to the plant pole. This placement will attenuate the open loop, plant pole's response. Finally the gain is varied until the closed loop responses are closely aligned with the naturally occurring open loop pole. The final closed loop poles are a set of complex poles and are located at $-7.55 \pm 3.64 i$. The frequency response can be evaluated using the bode plot in Figure \ref{IM_HeaveControlBode}. 
	 
	 \begin{figure}[H]
	 	\centering
	 	\includegraphics[height = 8cm]{../Design/Matlab/Controllers/heave_root.jpg}
	 	\caption{Heave Controller -  Root Locus}
	 	\label{IM_HeaveControlRoot}
	 \end{figure}
	 
	 The final cross over frequency is shown on the bode plot for the heave controller in Figure \ref{IM_HeaveControlBode}. The gain plot shows the controller adjust the crossover frequency to $7.99$\,rad/s which is close to the limit of the system. The controller also increases the phase of the system and has a final phase margin of $84$\textdegree.
	 
	 \begin{figure}[H]
	 	\centering
	 	\includegraphics[height = 8cm]{../Design/Matlab/Controllers/heave_bode.jpg}
	 	\caption{Heave Controller -  Bode Plots}
	 	\label{IM_HeaveControlBode}
	 \end{figure}
	 
	 An additional non linear element in the form of a limiter is brought into the system to limit the maximum and minimum thrust commands. The maximum limit is used to ensure the heave controller does not saturate the motors, thereby creating headroom for the angular rate controllers. The lower limit is used to ensure the vehicle always descends at a steady pace. The maximum thrust allowances are displayed in table \ref{tab:HeadRoomPercentages}, the final limits chosen are shown in Table \ref{tab:HeaveLimits}.
	 
	 \begin{table}[!]
	 	\centering
	 	\begin{tabular}{l | c | c |}
	 		Limit Name 				& Min & Max\\
	 		\hline\hline
	 		Integrator Wind Up 	   	& -1.5 	& 1.5 \\
	 		Thrust Command 		    & 10	& 48.64 \\
	 	\end{tabular}
	 	\caption{Heave Controller Limits}
	 	\label{tab:HeaveLimits}
	 \end{table}
	 
		 \subsubsection{Heave Controller Discussion}
		 Now that the system presents stable dynamic results in the frequency and Laplace domains, using the non-linear simulation, the time domain responses can be discussed in brief. The resultant step response, including the PI controller, is shown in Figure \ref{IM_HeaveStepDist}. To demonstrate the disturbance rejection capabilities of the design, a force of $10N$ is applied to the drone at $2.5s$.
		 
		 \begin{figure}[H]
		 	\centering
		 	\includegraphics[height = 8cm]{../Design/Matlab/Controllers/heave_step.jpg}
		 	\caption{Heave Controller -  Step Response}
		 	\label{IM_HeaveStepDist}
		 \end{figure}
		 
		 The heave controller, as the most inner loop, limits the response for the rest of the altitude control system. The proposed design brings the heave loop response close to the limits of the plant, thus producing a similar (but slower) timing constant to that of the motor-rotor system. The system reaches and settles within 5\% of the reference by $0.31s$, is critically damped and presents negligible overshoot. The system also shows to be capable of tracking an acceleration setpoint with zero steady state error. As shown in Figure \ref{IM_HeaveStepDist} the system can also respond quickly to a large, sudden and constant disturbance. The maximum rotor thrust commanded during this run is $3.35$\,N. \todo[inline]{How do I talk about gravity in this section? Can I just say gravity adds an offset of m*g and this plus the output must be less than the limit?}
	 
	 \subsection{Climb Rate Controller}
	 The climb rate controller is responsible for controlling the vertical velocity of the aircraft, in the earth frame. This introduces the need for rotating either the reference or the command into the body frame. The decision can be made by considering the frame in which the sensing information is provided. Most aircraft make use of some form of global positioning system and using differentiation can calculate speed. However, due to the application environment, this aircraft will most likely use a velocity measurement sensor relative to the body frame. The architecture for the climb rate controller is outlined in Figure \ref{IM_ClimbRateControlLoop}. 
	 
	 \begin{figure}[H]
	 	\centering
	 	\includegraphics[height = 3.75cm]{../References/Diagrams/ClimbRateLoop.jpg}
	 	\caption{Climb Rate Controller Closed Loop}
	 	\label{IM_ClimbRateControlLoop}
	 \end{figure}
	 
	 At near hover conditions the plant can be linearised and the rotation can be excluded. Figure \ref{IM_ClimbRateController} shows the simplified climb rate controller architecture. The computational time required for rotating the references can be considered by ensuring the controller design provides a reasonable phase margin. The speed of the climb rate controller is limited by the inner heave leave control loop. The controller must react quickly but there must still be a sufficient bandwidth ratio between the inner and outer loop. The controller must be able to track a setpoint with zero steady state error, but is not required to reject disturbances. The craft is required ot produce a steady approach to position targets, the climb rate controller should then exhibit a damped first order response.
	 
	 \begin{figure}[H]
	 	\centering
	 	\includegraphics[height = 3.5cm]{../References/Diagrams/ClimbRateController.jpg}
	 	\caption{Climb Rate Controller}
	 	\label{IM_ClimbRateController}
	 \end{figure}		
	 
	 The open loop poles of the climb rate system are located at the closed loop pole positions of the inner heave system, while the mathematical relationship between acceleration and velocity yields an additional open loop pole at the origin. The free integrator in the plant ensures the system will track a step response with zero steady state error while the proportional gain is used to speed up the system and achieve the desired bandwidth. The dynamic response of the system is evaluated using the root locus and bode plots shown in Figure \ref{IM_ClimbRateRoot} and \ref{IM_ClimbRateBode}.
	 
	 \begin{figure}[H]
	 	\centering
	 	\includegraphics[height = 8cm]{../Design/Matlab/Controllers/climb_rate_root.jpg}
	 	\caption{Climb Rate Controller -  Root Locus}
	 	\label{IM_ClimbRateRoot}
	 \end{figure}
	 
	 Figure \ref{IM_ClimbRateRoot} shows the location of the three final closed loop poles. There is a non dominant complex pair which is placed at $-5.11 \pm 3.76i$. The dominant pole is critically damped and located on the imaginary axis at $-4.89$.
	 
	 \begin{figure}[H]
	 	\centering
	 	\includegraphics[height = 8cm]{../Design/Matlab/Controllers/climb_rate_bode.jpg}
	 	\caption{Climb Rate Controller - Open-Loop Bode Plots}
	 	\label{IM_ClimbRateBode}
	 \end{figure}
	 
	 The bode plot shown in \ref{IM_ClimbRateBode} shows zero change in phase due to the controller architecture. The gain however is increased and moves the crossover frequency to $2.72$\,rad/s. The ratio of inner and outer loop crossover frequencies is then $2.91$, providing enough bandwidth between the inner and outer loops. The phase margin can then be calculated to be $72$\textdegree. 
	 
	 As shown in Figure \ref{IM_ClimbRateController} there is a limiter applied to the acceleration commands. This limit is present due to the confined operational environment and ensures that the climb rate controller does not saturate the horizontal velocity controllers. The final limits are shown in Table \ref{tab:ClimbrateLimits}
	 
	 \begin{table}[!]
	 	\centering
	 	\begin{tabular}{l | c | c |}
	 		Limit Name 						& Min & Max\\
	 		\hline\hline
	 		Acceleration Command 		    & -4 & 4 \\
	 	\end{tabular}
	 	\caption{Climb Rate Controller Limits}
	 	\label{tab:ClimbrateLimits}
	 \end{table}
	 
		 \subsubsection{Climb Rate Controller Discussion}
		 The dynamic response shows sufficient phase margin to handle unmodelled timing delays. While the ratio between the inner controller ensures this controller will not be influenced by the inner loop. The step response of the closed loop system is shown in Figure \ref{IM_ClimbRateStep}. The system has a $5$\% settling time of $0.747s$ and shows negligible overshoot. At $5s$ a disturbance of $10N$ is placed on the rotors and the system demonstrates the ability to continue tracking the desired setpoint with zero steady state error. \todo[inline]{Not sure if I need the disturbance rejection here, maybe I could show that it can't handle a measurement error in the inner loop?}
		 
		 \begin{figure}[H]
		 	\centering
		 	\includegraphics[height = 8cm]{../Design/Matlab/Controllers/climb_rate_step.jpg}
		 	\caption{Climb Rate Controller -  Step Response}
		 	\label{IM_ClimbRateStep}
		 \end{figure}
	 
	 \subsection{Altitude Hold Controller}
	 The final stage of the vertical control system is the altitude hold controller. This controller receives a desired altitude in the earth frame and outputs a reference velocity, also in the earth frame. The closed control loop block diagram is shown in Figure \ref{IM_AltHoldControlLoop}. 
	 
	 The altitude hold controller must be able to reject measurement errors in the inner loops, this can be achieved by adding an integrator into the controller. The system must also be able to track a set point with zero steady state error and must show a damped response with little overshoot. The system must be able to react quickly to commands, but is limited by the bandwidth of the climb rate system. The final bandwidth of this loop must be such that this controller is not influenced by the inner climb rate loop. As the most outer loop this system will have unmodelled errors, the controller must be robust and exhibit sufficient gain margin and a phase margin of at least $60$\textdegree.
	 
	 \begin{figure}[H]
	 	\centering
	 	\includegraphics[height = 4cm]{../References/Diagrams/AltHoldLoop.jpg}
	 	\caption{Altitude Hold Controller Closed Loop}
	 	\label{IM_AltHoldControlLoop}
	 \end{figure}
	 
	 The chosen controller architecture is shown in Figure \ref{IM_AltHoldController}. The proportional gain is used to vary the bandwidth to be within the limits of the system. An limited integrator is added to reject measurement errors in the inner loops, this component is represented by the faded integrator shown in Figure \ref{IM_AltHoldController} is not considered during linear analysis. The integrator shall be limited in such a way as to limit the interference of the proportional gain. This approach reduces the maximum disturbance rejection this controller can handle. To increase the bandwidth of the disturbance rejection capabilities, a PID controller architecture was also considered and the analysis was done for both control laws. \todo[inline]{Not sure if I should remove all the PID stuff here, I did consider it but the PID never stood a chance against the P and also do I need more disturbance rejection?}
	 
	 The system's dynamic response is analysed using the root locus shown in Figure \ref{IM_AltHoldRoot} and the bode plot shown in Figure \ref{IM_AltHoldBode}.
	 
	 \begin{figure}[H]
	 	\centering
	 	\includegraphics[height = 3.5cm]{../References/Diagrams/AltHoldController.jpg}
	 	\caption{Altitude Hold Controller}
	 	\label{IM_AltHoldController}
	 \end{figure}
	 
	 Figures \ref{IM_AltHoldRoot} and \ref{IM_AltHoldBode} evaluate a P controller against a PID controller. The dominant closed loop poles of the P controlled system are placed at $-2.03 \pm 0.58i$ and are slightly under damped with a damping ratio of $0.96$. 

	 The PID controller adds an additional pole and two additional zeros into the system. The closed loop poles of the PID controlled system are located at $-6.64$, $-3.59 \pm 5.03i$ and $-0.64 \pm 0.47i$. The two new zeros are placed at $-0.57$ and $-2.06$.
	 \todo[inline]{Must decide if I must keep the PID stuff in and evaluate properly here}
	 
	 \begin{figure}[H]
	 	\centering
	 	\includegraphics[height = 8cm]{../Design/Matlab/Controllers/altitude_root.jpg}
	 	\caption{Altitude Hold Controller -  Root Locus}
	 	\label{IM_AltHoldRoot}
	 \end{figure}
	 
	 The bode plot shows the PID controller producing a final cross over frequency of $1.90Rad/s$, this response is too fast for the inner climb rate system and will need to be redesigned or discarded. \todo[inline]{Must decide if I must keep the PID stuff in and evaluate properly here}. The P controller exhibits a cross over frequency of $0.91Rad/s$, this produces a ratio of $3.03$ between the inner and outer loops and a phase margin of $71.5$\textdegree. The phase of the system using a P controller crosses the $180$\textdegree  $\ $mark at $4.81Rad/s$ and has a gain margin of $18.7dB$.
	 
	 \begin{figure}[H]
	 	\centering
	 	\includegraphics[height = 8cm]{../Design/Matlab/Controllers/altitude_bode.jpg}
	 	\caption{Altitude Hold Controller -  Bode Plots}
	 	\label{IM_AltHoldBode}
	 \end{figure}
	 
	 To finalise the design, the two limiters are discussed. The first limiter is used to limit the effect of the integrator on the system as well as stop integrator wind up. The second limiter is used to limit the climb rate commands sent to the inner controllers. Both sets of limits are shown in Table \ref{tab:AltitudeControllerLimits}
	 
	 \begin{table}[!]
	 	\centering
	 	\begin{tabular}{l | c | c |}
	 		Limit Name 						& Min & Max\\
	 		\hline\hline
	 		Integrator Wind Up 				& -0.09 & 0.09 \\
	 		Climb Rate Command 		    	& -5 & 5 \\
	 	\end{tabular}
	 	\caption{Altitude Hold Controller Limits}
	 	\label{tab:AltitudeControllerLimits}
	 \end{table}
	 
		 \subsubsection{Altitude Hold Controller Discussion}
		 Although both the P and PID controllers exhibit stable dynamic responses, the PID controller exhibited too fast a response and will be influenced by the inner controllers. The system including only a P controller exhibits a step response as shown in Figure \ref{IM_AltHoldStep}, a disturbance of $10N$ is applied to the rotors at $10s$. The system has a $5\%$ settling time of $2.29s$ and tracks the set point with zero steady state error. The system handles the disturbance with a maximum overshoot of $0.01m$ and is critically damped.
		 
		 \begin{figure}[H]
		 	\centering
		 	\includegraphics[height = 8cm]{../Design/Matlab/Controllers/altitude_step_p_no_dist.jpg}
		 	\caption{Altitude Hold P Controller -  Step response}
		 	\label{IM_AltHoldStep}
		 \end{figure}
		 
		 However, if a measurement disturbance is present in the inner loops, this system will not track a setpoint with zero steady state error. To demonstrate this a constant offset of $0.05m/s$ is placed on the Z-Axis velocity measurement, Figure \ref{IM_AltHoldPDistStep} shows the current system cannot account for this disturbance. As shown in \ref{IM_AltHoldController}, a limited gain integrator is introduced into the system to help the system track steady state error. 
		 
		 \begin{figure}[H]
		 	\centering
		 	\includegraphics[height = 8cm]{../Design/Matlab/Controllers/altitude_step_p_dist.jpg}
		 	\caption{Altitude Hold P Controller -  Step response with inner loop measurement offset}
		 	\label{IM_AltHoldPDistStep}
		 \end{figure}
		 
		 The new controller must be limited in such a way as to exhibit a similar transient response as the existing P controller. Figure \ref{IM_AltHoldPIDistStep} shows the step response of the new system both with and without the $0.05m/s$ offset in the velocity measurement. As shown, the P controller including a limited I component introduces more overshoot into the system. The limits are designed to ensure the new controller introduces less than 10\% overshoot into the system. 
		 
		 \begin{figure}[H]
		 	\centering
		 	\includegraphics[height = 8cm]{../Design/Matlab/Controllers/altitude_step_pi.jpg}
		 	\caption{Altitude Hold P with Limited I Controller -  Step Responses}
		 	\label{IM_AltHoldPIDistStep}
		 \end{figure}
		 
		 \todo[inline]{Do I need to show impulse responses for altitude and climb rate? I mention the maximum thrust at the heave controller level, is that good enough?}
	 
\section{Horizontal Control}
This section describes the horizontal controller. This system is responsible for controlling the craft's North and East position in the earth frame, to do this the controller commands the pitch and roll rates. The narrow spaces in which the craft must fly means it is very important for the horizontal controller to respond quickly to commands and disturbances. The system must be fast but stay within it's thrust limits as to not affect the other controllers. The controller will also need a fast integrator term to reject disturbances.

The horizontal controller is designed as two sets of four cascaded control loops, one set for roll and one set for pitch. The most inner loop controls either the roll or pitch rate of the craft by commanding the virtual actuators $\delta_\phi$ and $\delta_\theta$ respectively. The desired angular rates are commanded by the tilt angle controller. The tilt angle controller is responsible for converting a desired translational acceleration into desired roll and pitch angles. The linear velocity controller receives it's setpoint from the most outer global position controller. The section begins by deriving the plant dynamics for roll and pitch.
	
	\subsection{Roll and Pitch Rate Dynamics}
	The roll and pitch acceleration dynamics can be derived using Newton mechanics at near hover conditions and the craft's inertia around the X-axis ($I_{xx}$) and the Y-axis ($I_{yy}$) respectively, the result is shown in \eqref{EQ_RollNewton} and \eqref{EQ_PitchNewton}.
	
	\begin{equation}
	\label{EQ_RollNewton}
	\dot{p} = \dfrac{L}{I_{xx}}
	\end{equation}
	
	\begin{equation}
	\label{EQ_PitchNewton}
	\dot{q} = \dfrac{M}{I_{yy}}
	\end{equation}
	
	The rotors introduce an additional timing delay into the dynamics, as shown in \eqref{EQ_MotorDelay}. The state space equation for both systems can be derived using the current angular rates ($p$ \& $q$) and the current angular moments ($L$ \& $M$) as the system states. The state space representation for roll is shown in \eqref{EQ_RollStateSpace1} and \eqref{EQ_RollStateSpace2}. The transfer function for roll acceleration can subsequently be calculated from the state space representation. Integrating the result produces the transfer function for roll rate as shown in \eqref{EQ_RollTF}. The same approach is followed for deriving the pitch rate dynamics shown in \ref{EQ_PitchTF}.
	
	\begin{eqnarray}
	\begin{bmatrix} \dot{L} \\ \dot{p}	\end{bmatrix}&=&\begin{bmatrix}\frac{1}{\tau}\ \ \ \ \ 0\\\frac{1}{I_{xx}} \ \ \ 0 \end{bmatrix} \begin{bmatrix} L \\ p \end{bmatrix} + \begin{bmatrix}\frac{1}{\tau}\\ 0 \end{bmatrix} \delta_\phi\label{EQ_RollStateSpace1}\\\label{EQ_RollStateSpace11} 
	y &=& \begin{bmatrix} 0 \ \ \ \ 1 \end{bmatrix} \begin{bmatrix} L \\ p \end{bmatrix} \label{EQ_RollStateSpace2}\\
	G(s)_{roll} &=& \frac{\frac{1}{\tau I_{xx}}}{s (s + \frac{1}{\tau})}\label{EQ_RollTF}\\
	G(s)_{pitch} &=& \frac{\frac{1}{\tau I_{yy}}}{s (s + \frac{1}{\tau})}\label{EQ_PitchTF}
	\end{eqnarray}
	
	The roll and pitch plants both have a naturally occurring integrator, an open loop pole at $-\dfrac{1}{\tau}$ and no naturally occurring zeros. As shown, the plant gain is inversely proportional to the specific axis inertia. The design of the craft creates a larger pitching plant gain than rolling plant gain. The design also gives the pitching moment a longer torque arm, creating a larger actuation torque. The controllers must compensate for this, resulting in similar closed loop system responses for both roll and pitch rate.
	
	\subsection{Roll and Pitch Rate Controllers}
	The roll and pitch rate controllers are the most inner loops of the horizontal controller and they command the $\delta_\phi$ and $\delta_\theta$ virtual actuators respectively. As the most inner controllers the outer loops of the horizontal controller are limited by the response and bandwidth of this system. The final system should have a first order response and must track a set point with zero steady state error.
	
	The horizontal controller requires fast disturbance rejection, to meet this requirements the roll and pitch rate controllers, as the fastest controllers, should include an integrator. To speed up the system a lead compensator is used however, the final commanded motor thrust must be validated against the limits provided in Table \ref{tab:HeadRoomPercentages}. The final controller is shown in Figure \ref{IM_RollRateController}. The controller gain values must be chosen such that the inner rate system is robust to unmodelled errors and have a sufficient gain and phase margin.
	
	\begin{figure}[H]
		\centering
		\includegraphics[height = 3.3cm]{../References/Diagrams/RollRateController.jpg}
		\caption{Roll and Pitch Rate Controller Design}
		\label{IM_RollRateController}
	\end{figure}	
	
	The dynamic response of the controlled system is evaluated using the root locus shown in Figure \ref{IM_RollRateControlRoot}. To maintain a first order response, the two dominant poles are kept close to the imaginary axis. The placement of the slower zero dictates how much influence the integrator can have on the system. The final placement of the slow integrator zero is $-1$\,rad/s.
	
	\begin{figure}[H]
		\centering
		\includegraphics[height = 8cm]{../Design/Matlab/Controllers/roll_rate_root_zoom.jpg}
		\caption{Roll Rate Controller -  Root Locus}
		\label{IM_RollRateControlRoot}
	\end{figure}
	
	The frequency response of the system is then investigated using the Bode plot shown in Figure \ref{IM_RollRateControlBode}. Unity feedback is compared against the chosen controller. As shown unity feedback produces a fast stable result, however there is an insufficient phase margin of $58$\textdegree\,in the system and no disturbance rejection. The controller adds an integrator to reject disturbances, this however reduces phase in the system. The lead compensator is then used to increase the phase and reach a final phase margin of $70$\textdegree\,with a crossover frequency of $4.67$\,rad/s.
	
	\begin{figure}[H]
		\centering
		\includegraphics[height = 8cm]{../Design/Matlab/Controllers/roll_rate_bode.jpg}
		\caption{Roll Rate Controller -  Bode Plot}
		\label{IM_RollRateControlBode}
	\end{figure}
	
		\subsubsection{Roll Rate Controller Discussion}
		The dynamic response of the system shows to be robust, damped and fast. The integrator term in the controller will ensure that the rate loop can handle disturbances. To stop integrator wind up, the controller includes a limiter on the integral term. The last consideration is the time domain response and the maximum impulse required of the motors. The step response of the system is shown in Figure \ref{IM_RollRateStep} \todo{Must I mention what the step value is?}. To simulate a disturbance, a $1$\,N loss in thrust is applied to both left hand motor numbers at $5$\,s.
		
		\begin{figure}[H]
			\centering
			\includegraphics[height = 8cm]{../Design/Matlab/Controllers/roll_rate_step.jpg}
			\caption{Roll Rate Controller -  Step Responses}
			\label{IM_RollRateStep}
		\end{figure}
		
		The system reaches the $5$\,\% settling time in $2$\,s and has an overshoot of $13$\,\%. Limiting the integral term can reduce the overshoot however, this will also limit the disturbance rejection capabilities of the system. The system handles the loss of thrust successfully and settles back within $5$\,\% of the setpoint in $2.5$\,s. The commands sent to the rotors are shown in \ref{IM_RollRateImpulse} with a maximum commanded thrust of $1.87$\,N.  
		
		\begin{figure}[H]
			\centering
			\includegraphics[height = 8cm]{../Design/Matlab/Controllers/roll_rate_impulse.jpg}
			\caption{Roll Rate Controller -  Step Responses}
			\label{IM_RollRateImpulse}
		\end{figure}
		
		\subsubsection{Pitch Rate Controller Discussion}
		The roll and pitch rate controllers both have the same architecture design shown in Figure \ref{IM_RollRateController}. The design of the craft however, means the pitch system will have a larger plant gain. It is desired that the roll and pitch rate systems have similar responses which means the pitch controller needs to have reduced gain compared to the roll controller. This unfortunately leaves the pitch controller to be more susceptible to disturbances. The flight strategy should take this into consideration and negate pitching disturbances as much as possible. Due to the longer torque arm, the pitching controller also commands a lower maximum thrust. The final step response of the pitch rate system is shown in Figure \ref{IM_PitchRateStep}
		
		\begin{figure}[H]
			\centering
			\includegraphics[height = 8cm]{../Design/Matlab/Controllers/pitch_rate_step.jpg}
			\caption{Pitch Rate Controller -  Step Responses}
			\label{IM_PitchRateStep}
		\end{figure}
		
	\subsection{Tilt Angle Controller}
	The tilt angle controller is responsible for controlling the desired roll and pitch angles of the craft. The controller does this by commanding angular rates it calculates from a translational acceleration reference in the earth frame. Using the current angular position, this earth frame reference is converted to the body frame and used to calculate the error in angular positions for roll and pitch. This error is then fed through a Proportional gain. This section makes reference to Figure \ref{IM_TiltAngleController} and begins by explaining the method used for converting the acceleration reference into desired angular rates.
	
	\begin{figure}[H]
		\centering
		\includegraphics[height = 6.5cm]{../References/Diagrams/TiltAngleController.jpg}
		\caption{Tilt Angle Controller}
		\label{IM_TiltAngleController}
	\end{figure}
		
		\subsubsection{Method of Conversion}	
		The first step to calculating the desired roll and pitch angles is to convert the earth frame set point into a body frame reference. The rotation matrix is calculated from the current Euler angles as seen in \eqref{EQ_RotationMatrix}.
		The desired body acceleration vector is compared with a unit body vector. To remove any dependency on yaw, a unit Z body vector is created, which is perfectly aligned with the Z-Axis and thrust generation of the craft. Utilising the dot product shown in \eqref{EQ_DotProduct} the magnitude of the rotation can be calculated. Unit vectors are used, so simply taking the arc cosine of the result will produce the magnitude of rotation. The axis of rotation can be calculated by using the cross product shown in \eqref{EQ_CrossProduct} and normalising the output. Figure \ref{IM_AngleMethod} is used a visual aid for the preceding description.
		
 		\begin{equation}
 		\label{EQ_DotProduct}
 		\vec{a} \, \bigcdot \, \vec{b} = |ab|\,\cos \alpha
 		\end{equation}
 		
 		\begin{equation}
 		\label{EQ_CrossProduct}
 		\vec{a} \, \times \, \vec{b} = \vec{c}
 		\end{equation}
		
		\begin{figure}[H]
			\centering
			\includegraphics[height = 10cm]{../References/Diagrams/ConversionMethod.jpg}
			\caption{Conversion Technique using Dot and Cross Products}
			\label{IM_AngleMethod}
		\end{figure}
	
		\subsubsection{Roll and Pitch Angle Controllers}
		The linear analysis is done by simplifying the system as shown in Figure \ref{IM_RollAngleLoop}.
		
		\begin{figure}[H]
			\centering
			\includegraphics[height = 3.5cm]{../References/Diagrams/RollAngleLoop.jpg}
			\caption{Roll Angle Closed Loop}
			\label{IM_RollAngleLoop}
		\end{figure}
		
		\begin{figure}[H]
			\centering
			\includegraphics[height = 8cm]{../Design/Matlab/Controllers/roll_angle_root.jpg}
			\caption{Roll Angle Controller -  Root Locus}
			\label{IM_RollAngleControlRoot}
		\end{figure}
		
		\begin{figure}[H]
			\centering
			\includegraphics[height = 8cm]{../Design/Matlab/Controllers/roll_angle_bode.jpg}
			\caption{Roll Angle Controller -  Bode Plots}
			\label{IM_RollAngleControlBode}
		\end{figure}
	
		\subsubsection{Tilt Angle Controller Discussion}
		
		\begin{figure}[H]
			\centering
			\includegraphics[height = 8cm]{../Design/Matlab/Controllers/roll_angle_step.jpg}
			\caption{Roll Angle Controller -  Step Responses}
			\label{IM_RollAngleStep}
		\end{figure}
		
		\begin{equation}
		\label{EQ_RollAngleTF}
		G(s) = \frac{\frac{1}{\tau I_{zz}}}{s (s + \frac{1}{\tau})}
		\end{equation}
	
	
	\subsection{Linear Velocity Control}
	
	\subsection{Global Position Tracking Control}
	
\section{Heading Controller}
This section describes the heading controller which is responsible for aligning the craft with a desired yaw angle reference. An angle reference is given to a yaw angle controller which outputs a yaw rate reference. The inner yaw rate controller commands the yawing moment around the Z-Axis of the craft. Both the vertical and the horizontal controllers have been designed to operate independently of the heading. The craft is however, expected to fly down a narrow channel. This calls for a method of aligning the body axis of the craft with a given heading in the earth frame. This heading could be given by a higher flight planning strategy.

The craft must be able to follow a heading setpoint with zero steady state error and have a reasonably damped response \todo[inline]{to vague?}. The heading controller must also be able to reject disturbances and will require an integrator in the control law. The yaw controller must also consider that the yaw torque generation has a reduction gain due to the lift to drag ratio. The design of the craft also means the yaw rate dynamics produce the lowest plant gain with the largest inertia component. The juxtaposition of a low actuation torque and lower plant gain leads to the heading controller typically being slower and exhibit less bandwidth when compared to the other controllers. Before a controller can be designed, the plant dynamics must be derived.
	
	\subsection{Yaw Rate Dynamics}
	Using Newton mechanics at near hover conditions, the yaw dynamics for the craft can be derived, the result is shown in equation \eqref{EQ_YawNewton}. $\dot{r}$ is the rotational acceleration of the craft and $N$ is the instantaneous moment experienced by the craft around the Z-Axis.
	
	\begin{equation}
	\label{EQ_YawNewton}
	\dot{r} = \dfrac{N}{I_{ZZ}}
	\end{equation}
	
	$\dot{r}$ is chosen as the output of the system with the state variable chosen as $N$. From this, the space equation for the system can be derived and is shown in \eqref{EQ_YawStateSpace1} and \eqref{EQ_YawStateSpace2}. 
	
	\begin{eqnarray}
	[\dot{N}] &=& - [\dfrac{1}{\tau}] \ [N] + [\dfrac{1}{\tau}] [\delta_\psi]\label{EQ_YawStateSpace1}\\\label{EQ_HeaveStateSpace22}
	[\dot{r}] &=& - [\dfrac{1}{I_{ZZ}}] \ [N]\label{EQ_YawStateSpace2}\\
	G(s)_{yaw} &=& \frac{\frac{1}{\tau I_{ZZ}}}{s (s + \frac{1}{\tau})}\label{EQ_YawTF}
	\end{eqnarray}
	
	From the state space representation, the transfer function for the yaw acceleration can be calculated. Integrating the result produces the transfer function for yaw rate, introducing a new pole into the system, the result is shown in \eqref{EQ_YawTF}. This plant now has two open loop poles, the first pole is due to the lag introduced by the motor rotor system, and lies at $\sigma = -\dfrac{1}{\tau} = 8$ with the second due to the integration of yaw acceleration to velocity. 
	
	\subsection{Yaw Rate Controller}	
	The yaw rate controller receives a yaw acceleration reference in radians per second (rad/s) and outputs the virtual actuator $\sigma_{\psi}$. The yawing moment is generated by air pressure on the rotors as they generate thrust, the reduction gain introduces the possibility of saturating the other controllers for large step inputs. However, as the most inner of the two heading loops, the yaw rate system limits the bandwidth of the outer yaw angle loop. The yaw rate controller must then produce enough bandwidth for the yaw angle controller, while ensuring it is not commanding thrust values above the limits described in \ref{tab:HeadRoomPercentages}. The yawing moment torque generation is also less accurately modelled \todo[inline]{better way of saying this? Trying to say that it is harder to accurately measure and is more susceptible for modelling errors} and the system must exhibit high stability with large gain and phase margins. The free integrator in the yaw rate system will produce zero steady state error. The design for this controller can then be designed as a simple P controller with a non-linear saturation as shown in Figure \ref{IM_YawRateController}.
	
	\begin{figure}[H]
		\centering
		\includegraphics[height = 3.5cm]{../References/Diagrams/YawRateController.jpg}
		\caption{Yaw Rate Controller -  Control Diagram}
		\label{IM_YawRateController}
	\end{figure}
	
	The dynamic response of the proportional (P) controller is evaluated using the root locus in Figure \ref{IM_YawRateControlRoot}. The controller adds no new poles or zeros. The proportional gain is used to move the closed loops and achieve the desired bandwidth. The final closed loops poles sit at $-4 \pm 1.93$\,rad/s, the poles are slightly under damped with a damping ratio of $0.9$.
	
	\begin{figure}[H]
		\centering
		\includegraphics[height = 8cm]{../Design/Matlab/Controllers/yaw_rate_root.jpg}
		\caption{Yaw Rate Controller -  Root Locus}
		\label{IM_YawRateControlRoot}
	\end{figure}
	
	The bode plot in Figure \ref{IM_YawRateControlBode} is used to evaluate the frequency response of the system against unity feedback. As expected the controller introduces no phase change into the system. Unity feedback produces a crossover frequency of $4.99$\,rad/s which is too fast for the heading system. Reducing the gain of the system increases the phase margin to $74$\textdegree and reduces the crossover frequency by nearly half to $2.37$\,rad/s.
	
	\begin{figure}[H]
		\centering
		\includegraphics[height = 8cm]{../Design/Matlab/Controllers/yaw_rate_bode.jpg}
		\caption{Yaw Rate Controller -  Bode Plots}
		\label{IM_YawRateControlBode}
	\end{figure}
	
		\subsubsection{Yaw Rate Controller Discussion}
		The controller increases stability in the system and produces the step response shown in Figure \ref{IM_YawRateStep}. The system has a $5\% $ settling time of $0.9s$ and negligible overshoot. The maximum thrust commanded per motor using this system is $0.26$\,N, which falls in the limits of this system. The low gain of the system makes this system susceptible to disturbances and calls for an outer angle loop. 
					
		\begin{figure}[H]
			\centering
			\includegraphics[height = 8cm]{../Design/Matlab/Controllers/yaw_rate_step.jpg}
			\caption{Yaw Rate Controller -  Step Responses}
			\label{IM_YawRateStep}
		\end{figure}
		
	\subsection{Yaw Angle Controller}	
	The yaw angle controller receives a heading reference in radians and outputs a yaw angle rate reference in radians per second to the inner rate controller as shown in Figure \ref{IM_YawAngleLoop}. This controller is limited by the inner loop and must ensure significant bandwidth between the inner and outer loops. The system must be able to reject disturbances and requires an integrator in the system. The system must be reasonably damped \todo[inline]{Too Vague?} with limited overshoot.

	\begin{figure}[H]
		\centering
		\includegraphics[height = 3.5cm]{../References/Diagrams/YawAngleLoop.jpg}
		\caption{Yaw Angle PI Controller -  Control Diagram}
		\label{IM_YawAngleLoop}
	\end{figure}
	
	Initially a Proportional Integral (PI) controller was considered as shown in Figure \ref{IM_YawAngleController}. The proportional gain is used to move the closed loop poles and achieve the desired bandwidth. The integral term is introduced to account for expected disturbances as well as measurement errors in the rate loop. The PI controller adds a new zero and a new pole into the system. 
	
	\begin{figure}[H]
		\centering
		\includegraphics[height = 3.5cm]{../References/Diagrams/YawAngleControllerPI.jpg}
		\caption{Yaw Angle PI Controller -  Control Diagram}
		\label{IM_YawAngleController}
	\end{figure}
	
	The second scheme was designed as a Proportional Integral Derivative (PID) controller as shown in Figure \ref{IM_YawAngleControllerPID}, the differential term is introduced to increase the phase of the system and adds an additional zero. The differential command is fed through a low pass filter to reduce noise on the command. As shown both controllers contain non linear elements that are not considered during the linear design. The components excluded during the analysis are all the limiters as well as the low pass filter seen in the differentiator portion of the PID leg.
	
	\begin{figure}[H]
		\centering
		\includegraphics[height = 5.5cm]{../References/Diagrams/YawAngleControllerPID.jpg}
		\caption{Yaw Angle PID Controller -  Control Diagram}
		\label{IM_YawAngleControllerPID}
	\end{figure}

	The dynamic response of each system can be evaluated using the root loci diagrams seen in Figure \ref{IM_YawAngleControlRoot}. The plant has two open loop poles at the closed loop yaw rate pole locations, as well as a new pole at the origin introduced by the mathematical relationship between speed and position. Both controllers introduce one new open loop pole at the origin and at least one zero. The PID controller introduces an additional zero into the system.  
	
	The final closed loop pole positions for the PI controller all lie on the imaginary axis and are located at $-3.62$, $-3.26$, $-0.88$ and $-0.25$. The PID controller has a dominant complex pair of closed loop poles at $-0.52 \pm 0.51 i$ with the other non-dominant closed loop pole pair sitting at $-3.48 \pm 2.54 i$. The PI controller has critically damped dominant poles where as the dominant poles for the PID controller are under damped with a damping ratio of $0.71$.
	
	\begin{figure}[H]
		\centering
		\includegraphics[height = 8cm]{../Design/Matlab/Controllers/yaw_angle_root.jpg}
		\caption{Yaw Angle Controller -  Root Locus (Left:PI, Right:PID)}
		\label{IM_YawAngleControlRoot}
	\end{figure}
	
	The frequency response of both systems is evaluated next. Using the open loop bode plots shown in Figure \ref{IM_YawAngleControlBode}, unity feedback is compared with a PI and PID controller. The first zero in both cases is placed close to the origin to limit the effect the new zero has on the system. The PID controller's second zero is placed to increase the phase of the system, allowing for more bandwidth in each leg of the controller. This additional phase allows for larger and more aggressive disturbance rejection, but will result in larger setpoints for the inner yaw rate loop.
	
	The gain of each system has similar bandwidth around the cross over point. The extra zero in the PID controller reduces the gradient of the gain slope off and increases the total phase of the system. The PID controller exhibits the second largest phase margin of $61$\textdegree\ which is found at $1.12$\,rad/s, the fastest of the three crossover frequencies. The PI controller achieves the desired bandwidth and has the slowest crossover frequency of $0.75$\,rad/s, this however relates to a lower phase margin of $59$\textdegree. The final crossover frequency of the yaw rate system was $2.37$\,rad/s, resulting in a ratio with the PID controller of $2.12$ and a ratio of $3.16$ with the PI controller. A larger ratio implies less risk of attenuation for the outer loop.
	
	\begin{figure}[H]
		\centering
		\includegraphics[height = 8cm]{../Design/Matlab/Controllers/yaw_angle_bode.jpg}
		\caption{Yaw Angle Controller -  Bode Plots}
		\label{IM_YawAngleControlBode}
	\end{figure}
	
	\subsubsection{Yaw Angle Controller Discussion}	
	Each controller was added to the non-linear simulation and evaluated in the time domain including the non linearities previously unconsidered. Figure \ref{IM_YawAngleStep} shows the results of both controllers with and without the limits as well as the additional low pass filter on the differential gain. The PID controller without any limits or low pass filtering had a $5$ \% settling time of $6.37s$, adding in the limits and filter decreases that time to $4.9s$. The PI controller had a settling time of $10.3s$ with no limits which was decreased to $8.8s$ by the addition of the limit on the integrator.
	
	All three systems exhibit some overshoot. The limiters must be carefully chosen to reduce overshoot while also still allowing for substantial disturbance rejection. The linear PI and PID systems produce overshoot of $17$\% and $29$\% respectively. The limits for the integrators on both systems was set to $\pm 0.1$\,rad/s. This limit also becomes the maximum offset this controller can successfully correct for. Both systems had significant overshoot reduction, the PI controller now only had an $12$\% overshoot, with the limited PID system showing only $6$\% overshoot. 
			
	\begin{figure}[H]
		\centering
		\includegraphics[height = 8cm]{../Design/Matlab/Controllers/yaw_angle_step_both_limits.jpg}
		\caption{Yaw Angle Controller -  Step Responses}
		\label{IM_YawAngleStep}
	\end{figure}
	
	As mentioned the limits introduce the maximum disturbance rejection capability of each system. Figure \ref{IM_YawAngleStepBoth} shows how each limited system handles a measurement offset of $0.1$\,rad/s in the yaw rate loop.
	
	\begin{figure}[H]
		\centering
		\includegraphics[height = 8cm]{../Design/Matlab/Controllers/yaw_angle_step_both_dist.jpg}
		\caption{Yaw Angle Controller -  Step Responses Including Inner Loop Measurement Offset}
		\label{IM_YawAngleStepBoth}
	\end{figure}
	
	The final consideration is the impulse each system creates for the yaw rate controller, Figure \ref{IM_YawAngleImpulse} demonstrates both limited systems impulse responses. As seen the PID controller commands a larger initial setpoint, more than double that of the PI controller.    
	
	\begin{figure}[H]
		\centering
		\includegraphics[height = 8cm]{../Design/Matlab/Controllers/yaw_angle_impulse.jpg}
		\caption{Yaw Angle Controller -  Impulses}
		\label{IM_YawAngleImpulse}
	\end{figure}
	
	The PI controller is chosen as the final controller implementation. 
	
	\section{Non-Linear Simulation}\label{SECT_Nonlinear}
	\subsection{Simulation Setup}
	\todo{Discuss plant and rotor motor stuff here}
	\subsubsection{Motor Mixer}		
	\todo{Show the maths behind the motor mixer}
	\begin{figure}[H]
		\centering
		\includegraphics[height = 5cm]{../References/Diagrams/MotorMixer.jpg}
		\caption{Motor Mixer}
		\label{IM_MotorMixer}
	\end{figure}	
	\subsubsection{Saturations}
	
	\begin{table}[!]
		\centering
		\begin{tabular}{l | c | c | c |}
			Parameter & Unit & Limit\\
			\hline\hline
			Thrust					& N 	& 50\\
			Moments					& Nm 	& 2\\
			Angular Rate 	   		& Rad/s & 1\\
			Lean Angle	    		& Rad 	& 0.35\\ %20 degrees
			Linear Velocity 	  	& m/s 	& 1\\
			Global Position  		& m 	& NA\\
		\end{tabular}
		\label{tab:UnitsLimits}
		\caption{Controlled Parameters Units and Limits}
	\end{table}
	\todo{Finalise table values}
	
	\subsubsection{Disturbances}
	\todo{Mention the disturbances added, reference to identification chapter and the order applied and what was being achieved}
	
	
	\subsection{Simulation Results}
	\todo{First show basic control and responses from each leg. Then chat about different disturbances and what happened}
		
		\section{Discussion}
		
		\section{Table}
		Platform Matrix
		\begin{table}[!]
			\centering
			\begin{tabular}{l | l | l | l | p{2cm} | p{2cm} |}
				Closed Loop & No. of Poles & No. of Zeros & Phase Margin (\textdegree) & Crossover Frequency (rad/s) & Settling Time - 5\% (S)\\
				\hline\hline
				Heave 	   				& 2 & 1 & 83.9556 & 7.9947 & 0.3722\\
				Climb Rate 			    & 3 & 4 & 8 & 5 & 3\\
				Altitude		 	  	& 3 & 7 & 5 & 6 & 5\\
				Control Algorithms  	& 4 & 5 & 4 & 6 & 3\\
				System Complexity 		& 3 & 2 & 5 & 7 & 2\\
				\hline\hline
				Total Score 			& 180 & 99 & 105 & 108 & 101\\
			\end{tabular}
			\label{tab:PlatformDesign}
			\caption{Rotor Configuration Scoring Matrix}
		\end{table}
		
\chapter{Obstacle Avoidance Implementation}
	
	The first section will discuss modelling this environment. the second section in this chapter will discuss the mathematical modelling of these sensors for the purpose of the simulation. 
	
	\section{Mathematical modelling}
		
		\subsection{World Creation}
		In order to correctly model the system, a thorough system identification needs to be completed. This entails real world measurements of the chosen platform. The methods and results from these experiments are covered in this section.
		
		\subsection{Sensor Implementation}	
		
		\subsection{Obstacle Distance Detection}
		
	\section{Virtual Spring Damper Model}
	
	\section{Obstacle Avoidance Controller}
		
	\section{Discussion}
	
		\subsection{Spring Damper Coefficients}
		
		\subsection{Effect of Varying Arm Length}
		
		\subsection{Limitations of proposed method}
\chapter{Simulated Flight Tests}
This chapter sets out to verify the performance of the proposed controller and flight strategies laid out in the previous sections of this report. The system identification chapter has outlined the data used as well as the configuration of the simulation. These parameters are used in conjunction with the controllers and flight strategy designed in previous chapters. 

The first section of this chapter specifies what the tests aim to demonstrate and is proceeded by the methodology followed to do so. Each simulated test is designed to showcase an aspect of the design. The results from the simulation test runs are presented and finally the feasibility and performance of the craft are discussed.
	
	\section{title}
	\begin{figure}[H]
		\centering
		\includegraphics[height = 6cm]{../References/Diagrams/Simulation.jpg} 
		\caption{High level view of simulation set up.}
		\label{IM_Sim}
		\end{figure}
		
	\section{Aims and Objectives}
	The objective of this work was to design an aerial vehicle that is capable of flight inside a confined space and narrow corridor. The first goal is to create a craft that is stable in the presence of system and sensor noise and impurities. The craft must be shown to to perform this task in the presence of disturbances. Once the craft is shown to produce stable results in the presence of disturbances using the designed flight controllers, the obstacle avoidance system needs to be tested. The first test must prove that in a simple environment with disturbances the system remains stable and ensures there is no collision. The next test must be to evaluate the drone navigating in a simple environment with the assistance of the obstacle avoidance controller. The difference with these tests will be to prove that the obstacle avoidance routine will navigate around obstacles and not simply just avoid them. After the drone can successfully navigate a simple environment, a more complex environment needs to be tested creating the needs for more complex manoeuvres. Finally the limitations in the proposed method need to be shown, an environment where the craft will not be able to complete it's desired mission must be shown to understand where improvements and future work can be aimed.

	\section{Testing Methodology and Results}
	This section describes the method and results used to achieve the testing aims outlined above. A multitude of test scenarios are generated, all with a specific purpose to achieve one of the testing aims described above. The testing is structured to follow a logical pattern of validations.
	
		\subsection{Fully Integrated and Controlled System}
		The objective of the first test is to validate the designed controllers in the presence of a large disturbance. The disturbance is simulated as a $5\pm1$\,m/s wind flowing at $45$\textdegree$\pm10$\textdegree, measured off the North axis in a counter clockwise direction. Figures \ref{IM_Test01} and \ref{IM_Test02} are basic step responses with the above mentioned disturbance. The angle of the wind is in the same direction as the East position reference and the opposite direction to the North position reference. The green line in both images depicts the undisturbed system, with the blue line showing the disturbed system.
		
		\begin{figure}[H]
			\centering
			\includegraphics[height = 8cm]{../References/Testing/SimpleWaypoint_Both_North.jpg}     
			\caption{Step response with disturbance - north position plot}
			\label{IM_Test01}
		\end{figure}
		
		\begin{figure}[H]
			\centering
			\includegraphics[height = 8cm]{../References/Testing/SimpleWaypoint_Both_East.jpg}     
			\caption{Step response with disturbance - east position plot}
			\label{IM_Test02}
		\end{figure}
		
		The wind causes the East controller to overshoot and sit around $1$\,m off the desired setpoint. The North controller is expected to fly into the wind and as expected exhibits a slower rise time to the non disturbed system.
				
		Next, the waypoint generator is loaded with five waypoints, commanding the drone to first reach a set height and then fly in a rectangle ending at the start point. The large wind is angled to push the drone South and West and is present from the beginning of the simulation. As the craft reaches it's set height the craft is commanded to maintain in it's current North and East position. The North and East controllers will have a small position error until the wind forces the drone off of the set point. An initial offset is thus expected before the velocity controllers are given larger setpoint to follow. Figures \ref{IM_Test11} and \ref{IM_Test12} show the North and East positions respectively.
		
		\begin{figure}[H]
			\centering
			\includegraphics[height = 8cm]{../References/Testing/SimpleWaypoint_5Wind_North.jpg}     
			\caption{Waypoint flight with disturbance - north position plot}
			\label{IM_Test11}
		\end{figure}
		
		The North position is initially offset by the wind and is commanded to fly against the wind before returning South with the wind pushing it towards the setpoint. The North position reaches within $0.5$\,m in both cases. The seemingly linear portion of the curve entails that the disturbance is causing the drone to hit the upper saturation limit of the velocity command.
		
		\begin{figure}[H]
			\centering
			\includegraphics[height = 8cm]{../References/Testing/SimpleWaypoint_5Wind_East.jpg}     
			\caption{Waypoint flight with disturbance - east position plot}
			\label{IM_Test12}
		\end{figure}
		
		The East position has more difficulty handling the disturbance. The non symmetry of the craft causes a larger surface area in that plane, which leads to a larger drag caused by wind disturbances. The East position is intially pushed a maximum of $1.5$\,m off of it's setpoint and settles to just less than a metre off the final desired position.
		
		The above tests show the craft to be stable under simulated disturbances. The craft has more difficulty rejecting disturbances caused by wind in the Y-Body frame of the craft and can be reduced by angling the craft towards the wind during flight.
					
		\subsection{Basic Obstacle Avoidance Routine}
		The next test is designed to evaluate the effectiveness of the obstacle avoidance routine in the presence of a disturbance. The Y-Body Axis was seen to be less resilient towards disturbances and will be tested for this purpose. The test is designed to see if the craft can maintain a set distance away from a wall while the tracking system and disturbance push it towards the wall.
		
		To accomplish a wall was placed along $1$\,m East and another wall at $-3$\,m East as shown by the red lines in Figure \ref{IM_Test21}. A constant $10$\,km/h wind was applied to the craft facing due East, pushing the craft into the wall. The Cyan Dot in the image represents the waypoint at (15, 0). There are two images in Figure \ref{IM_Test21}, the image on the left shows the flown flight path, while the image on the right adds the obstacle avoidance vector generated by the obstacle avoidance controller.
		
		\begin{figure}[H]
			\centering
			\includegraphics[height = 10cm]{../References/Testing/CorridorFlight_2Wind.jpg}     
			\caption{Corridor flight with disturbance - showing flight path (left) and obstacle avoidance vector (right)}
			\label{IM_Test21}
		\end{figure}
		
		The drone is commanded to position 0 East and the wind is attempting to force the Drone directly East as well. The obstacle avoidance vector successfully steers the craft away from the East wall and maintains an average distance of $0.62$\,m off of the wall with a standard deviation of $0.09$\,m. The obstacle avoidance vector shown in the coloured line of the right image shows the direction in which the drone is being pushed by the obstacle avoidance controller. The proximity to the East wall pushes the drone in a Westward direction, with a South facing component. The derivative portion of the controller creates the South facing vector as it moves in a Northerly direction.
		
		Figures \ref{IM_Test22}, \ref{IM_Test23} and \ref{IM_Test24} show the same flight with each position plotted against time. The green line shows the path flown in a scenario where the obstacle avoidance controller is deactivated. The red line is the set point with the blue line being the flown path with obstacle avoidance activated. The black dotted line in the East plot represent the wall at $1$\,m.
		
		\begin{figure}[H]
			\centering
			\includegraphics[height = 7.9cm]{../References/Testing/CorridorFlight_2Wind_North.jpg}     
			\caption{Corridor flight with disturbance - north position}
			\label{IM_Test22}
		\end{figure}
		
		The obstacle avoidance controller slows down the North settling time of the craft. The obstacle avoidance controller derivative component will increase damping in the system as seen by the reduced overshoot.
		
		\begin{figure}[H]
			\centering
			\includegraphics[height = 7.9cm]{../References/Testing/CorridorFlight_2Wind_East.jpg}     
			\caption{Corridor flight with disturbance - east position}
			\label{IM_Test23}
		\end{figure}
		
		As expected from the previous test, without the obstacle avoidance controller the drone will collide with the wall at $1$\,m. The obstacle avoidance controller pushes the craft off of it's setpoint to ensure it maintains a set distance form the wall. The increased damping in the obstacle avoidance controller also reduces the oscillations seen by the craft.
		
		\begin{figure}[H]
			\centering
			\includegraphics[height = 7.9cm]{../References/Testing/CorridorFlight_2Wind_Down.jpg}     
			\caption{Corridor flight with disturbance - down position}
			\label{IM_Test24}
		\end{figure}
		
		The horizontal obstacle avoidance has been the main focus this far. In this scenario a roof was placed at $6$\,m with a floor at $0$\,m. The craft settles an average distance of $0.71$\,m from the roof with a standard deviation of $0.03$\,m.
		
		The obstacle avoidance system has shown it's effectiveness by avoiding a collision in the presence of a disturbance. The routine successfully kept the drone away from the walls by maintaining a set distance. The routine also ensures that the craft flies slowly and steady when in the presence of obstacles.
		
		\subsection{Obstacle Avoidance Navigation}
		The next series of tests aims to test the capabilities of the obstacle avoidance routine and it's ability to navigate through a simple terrain. To test this different test scenarios have been configured. Each environment is in a sawtooth shape which traverses from South to North. The drone will be given only a North reference with the East position controller disabled. The obstacle avoidance routine will be required to ensure the drone does not collied with any of the walls, as well as find a path through the corridor to the desired North position. The test will be run three times. The first test will be in a wide corridor where the obstacle avoidance sensors will not always be activated. The second test will be a narrow corridor which will show how the craft behaves if it is constantly in close proximity to a wall. The third test will show the effect the yaw alignment algorithm, has on the system and vice versa.
		
			\subsubsection{Wide Corridor}
			The first test requires a large corridor to be used and are represented by the red lines in Figure \ref{IM_Test31}. The vehicle is commanded to fly to a North position of $34$\,m as shown by the cyan line. The flown path is shown in blue with the obstacle avoidance vector shown in the coloured line. The East position control is completely dictated by the obstacle avoidance routine.
			
			\begin{figure}[H]
				\centering
				\includegraphics[height = 14cm]{../References/Testing/WideCorridor3DProx.jpg}     
				\caption{Navigated flight in a wide corridor}
				\label{IM_Test31}
			\end{figure}
			
			The craft maintains a minimum distance of $0.5$\,m away from the wall and flies along an efficient path to reach the goal, only deferring from the straight line path when necessary to avoid a wall. Although a successful flight, the image does show the importance of the density for sensor placement. The sharp corners created by the wall only activate one of the sensors which creates a small obstacle avoidance vector. Having a more dense sensor placement will nudge the drone further away from the dangerous corners.
			
			\subsubsection{Narrow Corridor}
			The next tests showcases the drone's ability to fly in a confined space where the obstacle avoidance vector is always activated from multiple sides. The narrow corridor is shown in Figure \ref{IM_Test32}, with the drone requested to reach $17$\,m North. Once again the East position of the craft is completely dictated by the obstacle avoidance vector. The coloured line also once again demonstrates the current obstacle avoidance vector which is commanding velocities.
			
			\begin{figure}[H]
				\centering
				\includegraphics[height = 15cm]{../References/Testing/NarrowCorridor3DProx.jpg}     
				\caption{Navigated flight in a narrow corridor}
				\label{IM_Test32}
			\end{figure}
			
			The drone remains a minimum of $0.5$\,m away from the walls. This test demonstrates the importance of the derivative action of the obstacle avoidance vector, specifically in a confined space. Although all the sensors are active due to close proximity to multiple walls, the walls which the drone is approaching take precedence and the obstacle avoidance controller forces large action. The walls which the drone is not approaching have limited interference for the flight path.
			
			\subsubsection{Yaw Alignment in a Confined Space}
			The next test is designed to evaluate the operation of both the obstacle avoidance and the yaw alignment strategies when used in conjunction with one another. The wide corridor test is run again, this time with the yaw alignment routine activated. Figures \ref{IM_Test33}, \ref{IM_Test34} and \ref{IM_Test35} represent the results from those flights. The first image shows only the flown path with the second image including the heading vector of the craft.
			
			\begin{figure}[H]
				\centering
				\includegraphics[height = 10cm]{../References/Testing/WideCorridorYawAlign3D.jpg}     
				\caption{Navigated flight in wide corridor with yaw alignment}
				\label{IM_Test33}
			\end{figure}
			
			The path flown closely resembles that of the original path without yaw alignment. The craft does however get closer to the sharp corners of the wall, the yawing craft will affect the response of the obstacle avoidance controller due to the changing sensor readings. A tight sensor placement density will reduce the effect this has on the craft.
			
			\begin{figure}[H]
				\centering
				\includegraphics[height = 10cm]{../References/Testing/WideCorridorYawAlign3DAlign.jpg}     
				\caption{Navigated flight in wide corridor with yaw alignment showing current heading of craft}
				\label{IM_Test34}
			\end{figure}
			
			Figure \ref{IM_Test34} overlays the current heading of the craft on the flight path. The yaw alignment is shown to work reasonably well but can be quantified better using Figure \ref{IM_Test35} which shows the yaw error throughout the flight.
			
			\begin{figure}[H]
				\centering
				\includegraphics[height = 7.9cm]{../References/Testing/WideCorridorYawAlignYawError.jpg}     
				\caption{Yaw error of vehicle during navigated flight in wide corridor with yaw alignment}
				\label{IM_Test35}
			\end{figure}
			
			The yaw alignment routine runs at a slower pace to that of the velocity controller and hence has large error when there is a sudden change of direction. The average yaw alignment error for the flight is only $1.97$\textdegree with a large standard deviation of $21.53$\textdegree. The constant changes of the craft's velocity leads to varying alignment reference which can cause to a large deviation in error measurements. In some mission cases this might be unacceptable and the craft's current velocity should not be used as the alignment vector, but rather a set heading which can be varied.
		
		\subsection{Generic Mission Flight}
		Now that each component of the craft's control system has been tested and verified a generic example of flight test can be created. This group of tests is run in a simulated environment that contains a combination of open spaces, narrow corridors and unexpected obstacles. The waypoint generator is loaded with 9 waypoints loosely placed around the environment. The waypoints are placed to designed to make the craft explore the entire environment with the need for avoiding obstacles and flying down corridors un assisted. The final waypoint is placed in an unavailable location to assess the craft's behaviour. The test is run both with and without the yaw alignment routine enabled.
		
		The images seen in Figure \ref{IM_GenTest} show both runs with and without their proximity vectors. The waypoints are shown in cyan, starting at (0, 0) and moving in a counter clockwise direction. The last waypoint is placed inside the wall where the drone cannot fly. 
		
		\begin{figure}[H]
			\begin{tabular}{c c}
				\centering
				{\includegraphics[width = 3in]{../References/Testing/FullRun.jpg}} &
				{\includegraphics[width = 3in]{../References/Testing/FullRunAlign.jpg}}\\
				{\includegraphics[width = 3in]{../References/Testing/FullRunProx.jpg}} &
				{\includegraphics[width = 3in]{../References/Testing/FullRunAlignProx.jpg}}\\
			\end{tabular}
			\caption{Generic mission flight in a simulated environment with loosely placed waypoints. (Left - no yaw alignment, Right - yaw alignment)}
			\label{IM_GenTest}
		\end{figure}
		
		The craft does not hit any obstacles in both runs. A more detailed flight path or a more dense sensor placement would assist in ensuring the craft keeps a further proximity from the sharp corners.
		
%%		\texttt{\begin{figure}[H]
%%			\centering
%%			\includegraphics[height = 10cm]{../References/Testing/FullRun.jpg} 
%%			\caption{Navigated Flight in Narrow Corridor}
%%			\label{IM_Test41}
%%		\end{figure}
%%		
%%		\begin{figure}[H]
%%			\centering
%%			\includegraphics[height = 10cm]{../References/Testing/FullRunAlign.jpg} 
%%			\caption{Navigated Flight in Narrow Corridor}
%%			\label{IM_Test42}
%%		\end{figure}
%%		
%%		\begin{figure}[H]
%%			\centering
%%			\includegraphics[height = 10cm]{../References/Testing/FullRunProx.jpg} 
%%			\caption{Navigated Flight in Narrow Corridor}
%%			\label{IM_Test43}
%%		\end{figure}
%%		
%%		\begin{figure}[H]
%%			\centering
%%			\includegraphics[height = 10cm]{../References/Testing/FullRunAlignProx.jpg} 
%%			\caption{Navigated Flight in Narrow Corridor}
%%			\label{IM_Test44}
%%		\end{figure}}
		
		A better analysis of the separate runs can be made while observing the North and East positions relative to time as shown in Figures \ref{IM_Test46} and \ref{IM_Test47}. The green line shows the position without yaw alignment, while the blue line shows the position when yaw alignment is disabled. The average difference between the position along the North axis is $0.12$\,m with a standard deviation of $0.79$\,m. The East axis boasts a mean difference of only $3$\,cm with a standard deviation of $0.6$\,m.
		
		\begin{figure}[H]
			\centering
			\includegraphics[height = 7.9cm]{../References/Testing/FullRunBothNorth.jpg} 
			\caption{North position plot of a generic flight test. Showing the result both with and without yaw alignment.}
			\label{IM_Test46}
		\end{figure}
		
		\begin{figure}[H]
			\centering
			\includegraphics[height = 7.9cm]{../References/Testing/FullRunBothEast.jpg} 
			\caption{East position plot of a generic flight test. Showing the result both with and without yaw alignment.}
			\label{IM_Test47}
		\end{figure}
		
		To properly assess the yaw alignment strategy in this scenario Figures \ref{IM_Test45} and \ref{IM_Test48} are presented. Figure \ref{IM_Test45} overlays the current heading of the craft onto the flown flight path.
		
		\begin{figure}[H]
			\centering
			\includegraphics[height = 10cm]{../References/Testing/FullRunAlignYaw.jpg} 
			\caption{Yaw alignment plot of a generic flight test while utilising the heading alignment controller.}
			\label{IM_Test45}
		\end{figure}
		
		When the drone is constantly changing direction the yaw alignment has some difficulty maintaining the heading alignment. When there are longer stretches of straight path the yaw alignment has the time to reach the desired heading and maintain the yaw error within $3.56$\textdegree\, with a standard deviation of $5.49$\textdegree.
		Figure \ref{IM_Test48} gives a more empirical view of the yaw alignment by isolating the yaw error calculated.
		
		\begin{figure}[H]
			\centering
			\includegraphics[height = 7.9cm]{../References/Testing/FullRunAlignYawGraph.jpg} 
			\caption{Yaw error plot of a generic flight test while utilising the heading alignment controller.}
			\label{IM_Test48}
		\end{figure}
				
		\subsection{Limitations of the Design}
		The last set of tests are designed to show some of the limitations of the obstacle avoidance technique as a navigation algorithm. Situations exist where the obstacle avoidance routine will cause the drone to stall and not continue on it's mission. Two scenarios have been designed. The first scenario is a basic corner shown in Figure \ref{IM_Test51}.
		
		\begin{figure}[H]
			\centering
			\includegraphics[height = 11cm]{../References/Testing/Fail2.jpg} 
			\caption{Limitations of the obstacle avoidance routine as a navigation algorithm. Straight wall in a wide space.}
			\label{IM_Test51}
		\end{figure}
		
		As the craft approaches the wall at $5$\,m North it gets forced to stop by the obstacle avoidance vector. There is no additional information being fed to the the system that will inform it to continue on it's path. In this situation the obstacle avoidance works appropriately to avoid a collision. Figure \ref{IM_Test52} shows a situation where the drone will not collide with a wall, but the drone obstacle avoidance routine halts the mission unnecessarily.
		
		\begin{figure}[H]
			\centering
			\includegraphics[height = 9cm]{../References/Testing/Fail1.jpg} 
			\caption{Limitations of the obstacle avoidance routine as a navigation algorithm. Narrow corridor proceeding a wide open space.}
			\label{IM_Test52}
		\end{figure}
		
		The narrow corridor is wide enough for the drone to fit in, but the two front facing angled sensors pick up a disturbance and halt the drone from completing it's mission. Both of these situations call for a higher route planning algorithm, or more intelligently placed waypoints.

\chapter{Conclusions and Recommendations}
The final chapter concludes on the work done in this project. It begins by summarising the conclusions and discussions had during the course of this work and finalises by recommending future areas that should be focused on to achieve the goal flight inside a confined environment.

	\section{Summary and Conclusions}
	The thesis successfully designed and simulated a flight strategy capable of obstacle avoidance and basic navigation inside a confined environment. The strategy proposed was tested through simulation to maintain a distance of $0.5$\,m away from walls and obstacles while maintaining stable flight. The problem was solved by choosing a craft design to help fly in a narrow corridor and designing a set of controllers to maintain stable flight with an over arching flight strategy.
	
	The craft design of the vehicle was accomplished through analysis of conventional rotor wing configurations and flight theory. The vehicle required high payload capabilities for additional sensor packs and larger power sources to increase flight time. This was accomplished by choosing a design that optimised thrust capabilities in a narrow space. The design varies from a traditional quadcopter by having a $20\%$ overlap of the front and rear rotor sets.
	
	A three tiered controller system was designed to control the proposed platform in six degrees of freedom. The three tiers were broken into an altitude, horizontal and heading controller. All three systems were shown to be capable of rejecting disturbances and providing stable control. The altitude control system controls the height of the craft by commanding a climb rate which in turn controls the acceleration of the craft in line with the body Z-Axis. The horizontal controller is responsible for controlling the North and East position and velocity of the craft. This was accomplished by relating the North and East accelerations to relative pitch and roll angles for the craft which in turn command the pitch and roll angular rates. The heading controller is responsible for controlling the yaw angle of the craft by commanding a yaw rate. Each controller fed their setpoints into a motor mixer which created the correct thrust outputs for each motor.
	
	An investigation into existing collision avoidance techniques led to the successful generation of a proximity based obstacle avoidance routine. The method chosen requires a proximity measurement relative to the craft in the X, Y and Z-Body Axis and utilises the potential field method of obstacle avoidance. This allowed the craft to avoid obstacles by maintaining a set distance from obstructions in all three axes. 
	
	To enable autonomous flight a waypoint generator was created which enables the aircraft to automatically step between position set points. The environment and the design of craft required an additional flight strategy which aligns the heading of the craft with it's current direction. This ensured that the craft's longer axis is always in the direction of flight minimising drag and 
	proximity to narrow corridors.
	
	To validate the controller scheme and flight strategy, accurate mathematical modelling of the craft and disturbances was required. This was accomplished through a system identification process including real world measurements and collection of data from proposed sensors. The disturbances and the system were modelled using Matlab and Simulink.
		
	The simulation showed that the proposed flight strategy and controllers could be used for navigation in a confined environment. The platform was designed to ensure sufficient thrust capabilities for a larger power source and additional sensor payload making it suitable for expansion into industrial applications.
	
	\section{Recommendations}
	The following recommendations are proposed to improve the viability of the system as an autonomous platform and expand the work to create a real world implementation.
	
	\begin{itemize}
		\item The proposed obstacle avoidance routine has proven to be capable of providing navigation in some environments. The limitations of the design can be assisted by a higher level route planning algorithm which utilises sensor information to create more autonomy for missions.

		\item This work used simulation to prove the effectiveness of the system. To finalise the validation of the proposed system real world flight tests should be conducted using the proposed platform construction. It is recommended that prior to any implementation of the flight strategy or obstacle avoidance the mechanical construction is verified to be robust to ensure a good flight set up limiting risk during flight testing.
		
		\item Additional flight modes should be created to allow the pilot control of the more inner loops. This work creates a waypoint generator that feeds velocity commands. Situations exist, specifically during initial testing, that require the pilot is granted control of the craft's flight routine.
		
		\item Research has been done and reviewed to show the effects and subsequent disturbances of flight near walls. A more detailed measurement of these disturbances for the proposed craft should be done to ensure the system is capable of rejecting them substantially. 
		
		\item A robust state estimator would reduce noise and error on the measurements and allow for the implementation of a disturbance observer based control algorithm. Such an algorithm could assist with successful rejection of larger disturbances while limiting the effect on the tracking control.

	\end{itemize}

	
	

%\chapter{Removed}
\section{First}
\subsection{Operational Environment}
\todo[inline]{Need to decide if I should keep this in} 
Operating devices in hazardous environments are at times necessary and unavoidable. In the context of this paper a hazardous area is defined as an environment where there is a risk of fire or explosions due to the presence of sufficient quantities of flammable liquids, gases and dusts present in the atmosphere \cite{RockwellAutomation, STAHL}. There are two sets of area classifications, the International Electrotechnical Commission System for Certification to Standards  Relating to Equipment for Use in Explosive Atmospheres (IECEx) which was developed by North America. The European system gets it's name from a French term "Atmosph\`{e}res Explosives" and shall be dubbed ATEX \cite{ATEX, RockwellAutomation, STAHL}.

\subsubsection{IECEx Classification}
The IECEx system breaks up hazardous environments into different classes and divisions which are pertinent to the design of devices to be used in these regions. The varying definitions come with a set of different procedures and regulations that need to be adhered to. The North American classification system, has been designed to give a description of the possible quantities and type of volatile elements in the system \cite{RockwellAutomation, IECEx}.
 
\paragraph{Class Definition}
Three classes are defined and relate to the types of hazardous materials found in the environment. A Class 1 location is an area where flammable vapours and gases are present. Class 2 locations refer to the presence of flammable dusts, such as a coal mine. Finally a Class 3 location is defined as an area containing flammable fibres \cite{RockwellAutomation, IECEx}. 

\paragraph{Division Definition}
The division separation refers to the possibility of the substances, defined in the classes, to be present. Division 1 is defined as an area where the hazardous material will be present frequently. Division 2 states that during traditional operations there is less chance of the substance being found but  may become present through a fault \cite{RockwellAutomation}. 

The IECEx system further breaks down the classifications into groups but for the purpose of this paper it will be assumed that the device must operate in the most flammable gases.


\subsubsection{ATEX Classification}
Where the IECEx system breaks up the different areas into classes and then divisions, the ATEX system simply breaks them up into zones. These zones encompass the full detail of the frequency and the type of hazardous substances.

\paragraph{Zone Definitions}
There are six different zones, the first three zones all relate to the presence of a flammable gas or vapour. Zone 0 defines a gaseous atmosphere which is present continuously or for long periods. Zone 1 is an area where a dangerous cloud is likely to form during operations and Zone 2 is where an explosive atmosphere is not likely to occur and if it does will only be present for a short period of time. Zones 20, 21 and 22 are the designators for dust particles and have the same progression of frequency \cite{ATEX, SANS}.

\subsection{Gas and Temperature Groups}
Both the ATEX and the IECEx classification systems can be further broken down into different gas groups as well as temperature classes. The gas groups are defined by the explosive properties of the materials and are shown in the table below.

The temperature classes are determined according to the maximum allowable surface temperature before an ignition is caused. Both classification systems use the same limits to separate the classes, the difference being that the American system breaks up the classes into sub-classes for more specific definitions.
The ATEX systems varies from a maximum allowable temperature of 450\textdegree (Class T1) to a maximum allowable temperature of 85\textdegree (Class T6) \cite{ATEX, STAHL, SANS}.

\begin{table}[!]
\centering
\begin{tabular}{l | c | c}

Gas Element & IECEx Group & ATEX Group\\
\hline\hline
Acetylene & A & II C\\
Hydrogen & B & II C\\
Ethylene & C & II B\\
Propane & D &  II A\\

\end{tabular}

\caption{Gas groups according to classification systems}
\end{table}

\paragraph{Equipment Categories}
The ATEX definitions also include corresponding equipment categories. Once the device is certified under a specific category, it can be utilised in that category's prescribed zones. Category 1 equipment is considered the most protected and safe devices, they may operate in Zone 0 and Zone 20 and all lower rated zones. Category 2 is the second most rigid equipment group and can be utilised in Zone 1 and Zone 21, as well as the lower zones. Category 3  equipment may only be used in Zone 2 and Zone 22 \cite{ATEX, SANS}.



\subsubsection{Effects of Abnormal Atmospheric Conditions}
Due to the environment in the above mentioned hazardous areas containing mixtures of gases, some atmospheric properties differ from regular air. These conditions could affect the technical operations of certain devices \cite{HC} and for aerial vehicles it is extremely important that the designer has an understanding of the environment the rotors will be flying in \cite{Leishman}. 



%RESEARCH INTO INTRISIC SAFETY
\subsection{Designing for Hazardous Locations}
When designing for hazardous and volatile environments there are stringent standards that need to be followed. The classifications described above determine the level of protection needed in the devices. The main cause of concern in these areas is the generation of fire or equivalent ignition sources, which could cause an explosion.
There are numerous amounts of methods to design in volatile environments. Each of which comes with a set of standards defined by The International Organisation of Standards (ISO). In South Africa the set of standards used are created by the South African Bureau of Standards (SABS) and is documented in a South African National Standards (SANS) document.

\subsubsection{Explosions}
\cite{RockwellAutomation, STAHL}
\paragraph{Causes}
\paragraph{Control}

\subsubsection{Flame/Explosion Proof Enclosures}
A flame proof enclosure is defined by the South African National Standards (SANS) as an enclosure that contains parts which could cause an ignition. The casing must be able to withstand the pressure created by an explosion and not allow the energy to escape and create a further reaction with the explosive atmosphere \cite{FProof}. Each door or cover into the enclosure needs to be accessible only through the manipulation of a threaded fastener. 

\paragraph{Interconnecting Joints}
Spigot Joints
Serrated Joints
Threaded Joints 
Gaskets, O Rings and seals

\paragraph{Operating/Rotating Shafts}
Provision for wear and tear (Gap enlargement)
Cylindrical Joint, labyrinth joint
Gap Diagram

\paragraph{Bearings}
at least one element must be non sparking
Sleeve bearing not permitted for $\mathrm{II}$C
Rolling-element


\subsubsection{Encapsulation}
\cite{Encaps}

\subsubsection{Intrinsic Safety}
\cite{Insafe}


       












%\chapter{Templates}

\section{Table}
Platform Matrix
\begin{table}[!]
	\centering
	\begin{tabular}{l | c | c | c | c | c | c |}
		Factor & Weighting & Traditional & Co Axial & Tandem & Multirotor & Hybrid\\
		\hline\hline
		Hover efficiency 	   	& 5 & 8 & 7 & 6 & 3 & 2\\
		Physical Size 		    & 3 & 4 & 8 & 5 & 3 & 5\\
		Manoeuvrability 	  	& 3 & 7 & 5 & 6 & 9 & 5\\
		Control Algorithms  	& 4 & 5 & 4 & 6 & 8 & 3\\
		System Complexity 		& 3 & 2 & 5 & 7 & 6 & 2\\
		\hline\hline
		Total Score & 180 & 99 & 105 & 108 & 101 & 58\\
	\end{tabular}
	\label{TAB_PlatformDesign}
	\caption{Rotor Configuration Scoring Matrix}
\end{table}


\section{Figure}
\begin{figure}[H]
\centering
\includegraphics[height = 6cm]{Images/Litreature/TiltRotor}     
\caption{Hager's design for a telescopic tilt rotor system (Taken from \cite{Heli})}
\label{IM_EG}
\end{figure}




\begin{enumerate}
\item Flight time and efficiency
\item Geometry and size
\item Drone Manoeuvrability
\item Control algorithms
\item Mechanical complexity
\end{enumerate}

\begin{equation}
\label{EQ_EG}
DL (\frac{N}{m^{2}})= \frac{T}{A} = \frac{1}{2} \rho v_\infty^2
\end{equation}


%*******************************BIBLIOGRAPHY COMMANDS*******************************%

\bibliographystyle{plain}
\bibliography{Angus_Steele,Masters}

\end{document}   

