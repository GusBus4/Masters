\chapter{Removed}
\section{First}
\subsection{Operational Environment}
\todo[inline]{Need to decide if I should keep this in} 
Operating devices in hazardous environments are at times necessary and unavoidable. In the context of this paper a hazardous area is defined as an environment where there is a risk of fire or explosions due to the presence of sufficient quantities of flammable liquids, gases and dusts present in the atmosphere \cite{RockwellAutomation, STAHL}. There are two sets of area classifications, the International Electrotechnical Commission System for Certification to Standards  Relating to Equipment for Use in Explosive Atmospheres (IECEx) which was developed by North America. The European system gets it's name from a French term "Atmosph\`{e}res Explosives" and shall be dubbed ATEX \cite{ATEX, RockwellAutomation, STAHL}.

\subsubsection{IECEx Classification}
The IECEx system breaks up hazardous environments into different classes and divisions which are pertinent to the design of devices to be used in these regions. The varying definitions come with a set of different procedures and regulations that need to be adhered to. The North American classification system, has been designed to give a description of the possible quantities and type of volatile elements in the system \cite{RockwellAutomation, IECEx}.
 
\paragraph{Class Definition}
Three classes are defined and relate to the types of hazardous materials found in the environment. A Class 1 location is an area where flammable vapours and gases are present. Class 2 locations refer to the presence of flammable dusts, such as a coal mine. Finally a Class 3 location is defined as an area containing flammable fibres \cite{RockwellAutomation, IECEx}. 

\paragraph{Division Definition}
The division separation refers to the possibility of the substances, defined in the classes, to be present. Division 1 is defined as an area where the hazardous material will be present frequently. Division 2 states that during traditional operations there is less chance of the substance being found but  may become present through a fault \cite{RockwellAutomation}. 

The IECEx system further breaks down the classifications into groups but for the purpose of this paper it will be assumed that the device must operate in the most flammable gases.


\subsubsection{ATEX Classification}
Where the IECEx system breaks up the different areas into classes and then divisions, the ATEX system simply breaks them up into zones. These zones encompass the full detail of the frequency and the type of hazardous substances.

\paragraph{Zone Definitions}
There are six different zones, the first three zones all relate to the presence of a flammable gas or vapour. Zone 0 defines a gaseous atmosphere which is present continuously or for long periods. Zone 1 is an area where a dangerous cloud is likely to form during operations and Zone 2 is where an explosive atmosphere is not likely to occur and if it does will only be present for a short period of time. Zones 20, 21 and 22 are the designators for dust particles and have the same progression of frequency \cite{ATEX, SANS}.

\subsection{Gas and Temperature Groups}
Both the ATEX and the IECEx classification systems can be further broken down into different gas groups as well as temperature classes. The gas groups are defined by the explosive properties of the materials and are shown in the table below.

The temperature classes are determined according to the maximum allowable surface temperature before an ignition is caused. Both classification systems use the same limits to separate the classes, the difference being that the American system breaks up the classes into sub-classes for more specific definitions.
The ATEX systems varies from a maximum allowable temperature of 450\textdegree (Class T1) to a maximum allowable temperature of 85\textdegree (Class T6) \cite{ATEX, STAHL, SANS}.

\begin{table}[!]
\centering
\begin{tabular}{l | c | c}

Gas Element & IECEx Group & ATEX Group\\
\hline\hline
Acetylene & A & II C\\
Hydrogen & B & II C\\
Ethylene & C & II B\\
Propane & D &  II A\\

\end{tabular}

\caption{Gas groups according to classification systems}
\end{table}

\paragraph{Equipment Categories}
The ATEX definitions also include corresponding equipment categories. Once the device is certified under a specific category, it can be utilised in that category's prescribed zones. Category 1 equipment is considered the most protected and safe devices, they may operate in Zone 0 and Zone 20 and all lower rated zones. Category 2 is the second most rigid equipment group and can be utilised in Zone 1 and Zone 21, as well as the lower zones. Category 3  equipment may only be used in Zone 2 and Zone 22 \cite{ATEX, SANS}.



\subsubsection{Effects of Abnormal Atmospheric Conditions}
Due to the environment in the above mentioned hazardous areas containing mixtures of gases, some atmospheric properties differ from regular air. These conditions could affect the technical operations of certain devices \cite{HC} and for aerial vehicles it is extremely important that the designer has an understanding of the environment the rotors will be flying in \cite{Leishman}. 



%RESEARCH INTO INTRISIC SAFETY
\subsection{Designing for Hazardous Locations}
When designing for hazardous and volatile environments there are stringent standards that need to be followed. The classifications described above determine the level of protection needed in the devices. The main cause of concern in these areas is the generation of fire or equivalent ignition sources, which could cause an explosion.
There are numerous amounts of methods to design in volatile environments. Each of which comes with a set of standards defined by The International Organisation of Standards (ISO). In South Africa the set of standards used are created by the South African Bureau of Standards (SABS) and is documented in a South African National Standards (SANS) document.

\subsubsection{Explosions}
\cite{RockwellAutomation, STAHL}
\paragraph{Causes}
\paragraph{Control}

\subsubsection{Flame/Explosion Proof Enclosures}
A flame proof enclosure is defined by the South African National Standards (SANS) as an enclosure that contains parts which could cause an ignition. The casing must be able to withstand the pressure created by an explosion and not allow the energy to escape and create a further reaction with the explosive atmosphere \cite{FProof}. Each door or cover into the enclosure needs to be accessible only through the manipulation of a threaded fastener. 

\paragraph{Interconnecting Joints}
Spigot Joints
Serrated Joints
Threaded Joints 
Gaskets, O Rings and seals

\paragraph{Operating/Rotating Shafts}
Provision for wear and tear (Gap enlargement)
Cylindrical Joint, labyrinth joint
Gap Diagram

\paragraph{Bearings}
at least one element must be non sparking
Sleeve bearing not permitted for $\mathrm{II}$C
Rolling-element


\subsubsection{Encapsulation}
\cite{Encaps}

\subsubsection{Intrinsic Safety}
\cite{Insafe}


       











